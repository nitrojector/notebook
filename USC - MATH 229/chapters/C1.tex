\chapter{Vector and the Geometry of Space}

\section{3-Dimensional Space}

\subsection{2D Coordinates}

\begin{equation}
	\mathbb{R}^2 = \left\{(x,y) \mid x,y \in \mathbb{R}\right\}
\end{equation}

\subsection{3D Coordinates}

\begin{equation}
	\mathbb{R}^3 = \left\{(x,y,z) \mid x,y,z \in \mathbb{R}\right\}
\end{equation}

\begin{lemma}[Distance Between 2 Points]
	\begin{equation}
		\abs{P_1 P_2} = \sqrt{(x_1 - x_2)^2 + (y_1 - y_2)^2 + (z_1 - z_2)^2}
	\end{equation}
\end{lemma}

\begin{proof}
	Easily proven by using the Pythagorean Theorem twice.
\end{proof}

\begin{lemma}[Spherical Surface]
	Given point $C(a,b,c)$ and $P(x,y,z)$ where $P$ is a point on the spherical surface and $r$ is the radius of the sphere.
	
	\begin{equation}
		(x - a)^2 + (y - b)^2 + (z - c)^2 = r^2
	\end{equation}

	To define a solid spherical space

	\begin{equation}
		\sqrt{(x - a)^2 + (y - b)^2 + (z - c)^2} \leq r
	\end{equation}
\end{lemma}

\section{Vectors}

\begin{definition}[Vector]
	Vector is a quantity that has a \textbf{magnitude} and a \textbf{direction}.
\end{definition}

We say that two vectors $\vec{u}$ and $\vec{v}$ are equal if they have the same length and direction.

\subsection{Vector Operation}

Omitted

\subsection{Components}

In $\mathbb{R}^2$

\begin{equation}
	\vec{a} \equiv \langle a_1, a_2 \rangle
\end{equation}

In $\mathbb{R}^3$

\begin{equation}
	\begin{cases}
		\vec{a} &\equiv \langle a_1, a_2, a_3 \rangle\\
		\vec{0} &\equiv \langle 0, 0, 0, \rangle
	\end{cases}
\end{equation}

\begin{definition}
	Length of $\vec{a} \equiv \langle a_1, a_2, a_3 \rangle$ is

	\begin{equation}
		\abs{\vec{a}} = \sqrt{{a_1}^2 + {a_2}^2 + {a_3}^2}
	\end{equation}
\end{definition}

\subsection{Standard Basis Vectors}

\begin{equation}
	\begin{cases}
		\hat{i} &= \langle 1, 0, 0 \rangle\\
		\hat{j} &= \langle 0, 1, 0 \rangle\\
		\hat{k} &= \langle 0, 0, 1 \rangle\\
	\end{cases}
\end{equation}

\section{The Dot Products}

\begin{definition}
	\begin{equation}
		\vec a = \langle a_1, a_2, a_3 \rangle \qquad \vec b = \langle b_1, b_2, b_3 \rangle
	\end{equation}

	Then, the dot product is

	\begin{equation}
		\vec a \cdot \vec b \equiv a_1 b_1 + a_2 b_2 + a_3 b_3 
	\end{equation}
\end{definition}

\textbf{Properties}

\begin{enumerate}[1.]
	\item $\vec a \cdot \vec a = {a_1}^2 + {a_2}^2 + {a_3}^2 = \abs{\vec a}^2$
	\item $\vec a \cdot \vec b = \vec b \cdot \vec a$
	\item $\vec a \cdot (\vec b + \vec c) = \vec a \cdot \vec b + \vec a \cdot \vec c$
	\item $(c\vec a) \cdot \vec b = c(\vec a \cdot \vec b)$
	\item $\vec 0 \cdot \vec a = 0$
\end{enumerate}

\begin{theorem}
	\begin{align}
		\vec a \cdot \vec b &= \abs{\vec a} \abs{\vec b} \cos\theta\\
		\cos\theta &= \frac{\vec a \cdot \vec b}{\abs{\vec a} \abs{\vec b}}, 0 \leq \theta \leq \pi
	\end{align}
\end{theorem}

\begin{lemma}
	\begin{itemize}
		\item If $\vec a \cdot \vec b > 0$ then $\cos \theta > 0 \implies \theta < \frac{\pi}{2}$
		\item If $\vec a \cdot \vec b < 0$ then $\cos \theta < 0 \implies \theta > \frac{\pi}{2}$
		\item If $\vec a \cdot \vec b = 0$, then $\theta = \frac{\pi}{2}, \vec a \perp \vec b$
	\end{itemize}
\end{lemma}

\subsection{Law of Cosine}

\begin{equation}
	\abs{\vec a - \vec b}^2 = \abs{\vec a}^2 + \abs{\vec b}^2 - 2 \abs{\vec a}\abs{\vec b}\cos\theta
\end{equation}

\begin{proof}
	\begin{align}
		\abs{\vec a - \vec b}^2 &= (\vec a - \vec b) \cdot (\vec a - \vec b)\\
		&= \abs{\vec a}^2 - 2 \vec a \cdot \vec b + \abs{\vec b}^2\\
		&= \abs{\vec a}^2 + \abs{\vec b}^2 - 2ab\cos(\theta)
	\end{align}	
\end{proof}

\subsection{Projection}
\begin{figure}[H]
	\centering
		\begin{tikzpicture}[dot/.style={circle,inner sep=1pt,fill,label={#1},name=#1},
			extended line/.style={shorten >=-#1,shorten <=-#1},
			extended line/.default=1cm]
			\draw[thick,-stealth] (-4.5,0) -- (4.5,0);
			\draw[thick,-stealth] (0,0) -- (0,4.5);
			\coordinate (A) at (0,0);
			\coordinate (B) at (-4,3);
			\draw [extended line=0.5cm, stealth-stealth] (A) -- (B) node[pos=1.15,font=\small]{$\vec a$};     
			\draw [ -stealth] (0,0) -- (-2.6, 4.3) coordinate (yn) node[right]{$\,\vec b$}; 
			\draw[dashed] (yn) --  node[midway,above left]{$\varepsilon$} ($(A)!(yn)!(B)$) node[below left]{$\vec b_p$};
		\end{tikzpicture}  
	\caption{Projection}
	\label{fig:projection}
\end{figure}

\textbf{Add to this.}

\begin{align}
	\abs{\vec b}
\end{align}

\begin{example}
	\begin{equation}
		\vec u = \langle 1,1,2\rangle \qquad \vec v = \langle -2,3,1\rangle
	\end{equation}

	Find projection of $\vec u$ onto $\vec v$
\end{example}

\begin{sol}
	\begin{align}
		\mathrm{comp}_{\vec c} \vec u &= \vec u \cdot \frac{\vec v}{\abs{\vec v}}\\
		&= \frac{-2 + 3 + 2}{\sqrt{14}} = \frac{3}{\sqrt{14}}
	\end{align}

	\begin{equation}
		\mathrm{proj}_{\vec v} \vec u = \left(\mathrm{comp}_{\vec v} \vec u\right) \frac{\vec v}{\abs{\vec v}} = \frac{3}{\sqrt{14}} \cdot \frac{\vec v}{\sqrt{v}} = \frac{3}{14} \vec v
	\end{equation}
\end{sol}

\subsection{Work}

Move an an object from $P$ to $Q$ with a force $\vec F$ forming an angle $\theta$ with the displacement vector $\vec D$.

\begin{align}
	\text{Work} &\equiv \text{Force} \cross \text{Dist}\\
	W &= (\abs{\vec F}\cos \theta) \abs{\vec D}\\
	&= \abs{\vec F}\abs{\vec D}\cos \theta\\
	&= \vec F \cdot \vec D\\
	&\implies W = \vec F \cdot \vec D
\end{align}

\begin{example}
	Move a particle from $P(2,1,0)[m]$ to $Q(4,6,2)$ with a force $\vec F = \langle 3, 4, 5 \langle [N]$.

	What is the work done by $\vec F$?
\end{example}

\begin{sol}
	\begin{align}
		W &= \vec F \cdot \vec{PQ}\\
		&= \langle 3, 4, 5 \rangle \cdot \langle 2, 5, 2\rangle\\
		&= \SI{36}{\N\m}
	\end{align}
\end{sol}

\section{The Cross Product}

\begin{definition}
	Given the vectors

	\begin{equation}
		\vec a = \langle a_1, a_2, a_3 \rangle, \vec b = \langle b_1, b_2, b_3 \rangle
	\end{equation}

	The cross product is defined as

	\begin{equation}
		\vec a \cross \vec b = \begin{vmatrix}
			\hat i & \hat j & \hat k\\
			a_1 & a_2 & a_3\\
			b_1 & b_2 & b_3
		\end{vmatrix} = \langle a_2 b_3 - a_3 b_2, a_3 b_1 - a_1 b_3, a_1 b_2 - a_2 b_1 \rangle
	\end{equation}
\end{definition}

\textbf{Properties of the Dot Product}

\begin{enumerate}[1.]
	\item $(\vec a \cross \vec b) \perp \vec a \& \vec b$ and the direction follows the right-hand rule.
	\item $\abs{\vec a \cross \vec b} = \abs{\vec a} \abs{\vec b} \sin\theta, 0 \leq \theta \leq \pi$
	\item $\abs{\vec a \cross \vec b} = $ the area of the parallelogram formed by the two vectors.
	\item If $\vec a \parallel \vec b$, then $\vec a \cross \vec b = \vec 0$
	\item Cross product of basis vectors
	
	\begin{equation}
		\begin{cases}
			\hat i \cross \hat j &= \hat k\\
			\hat j \cross \hat k &= \hat i\\
			\hat k \cross \hat i &= \hat j
		\end{cases}
	\end{equation}

	\item The cross product is not commutative
	\item The cross product is not associative

	\begin{example}
		\begin{equation}
			\begin{cases}
				\hat i \cross (\hat i \cross \hat j) &= \hat i \cross \hat k = -\hat j\\
				(\hat i \cross \hat i) \cross \hat j &= \vec 0 \cross \hat j = \vec 0
			\end{cases}
		\end{equation}
	\end{example}

	\item You can find the normal vector to a plane by applying the cross product to two non-parallel vectors on that plane.
\end{enumerate}

\begin{example}
	Given points

	\begin{equation*}
		P(1,4,6), Q(-2,5,1), R(1,-1,1)
	\end{equation*}

	that lie on a plane

	\begin{enumerate}[a)]
		\item Find the vector normal to the plane
		\item Find the area of $\triangle{PQR}$
	\end{enumerate}
\end{example}

\begin{sol}
	TBA
\end{sol}

\begin{definition}[Triple Products]
	\begin{equation} \label{eq:scalar-triple}
		\vec a \cdot (\vec b \cross \vec c) = \vec b \cdot (\vec c \cross \vec a) = \vec c \cdot (\vec a \cross \vec b)
	\end{equation}

	\cref*{eq:scalar-triple} shows the scalar triple product. This is also the volume of the parallelepiped.

	\begin{equation} \label{eq:vector-triple}
		\vec a \cross (\vec b \cross \vec c) = (\vec a \cdot \vec c) \vec b - (\vec a \cdot \vec b) \vec c
	\end{equation}

	\cref*{eq:vector-triple} shows the vector triple product.
\end{definition}

\begin{lemma}
	If $\vec a, \vec b$, and $\vec c$ are on the same plane (\textit{coplanar}), then $\vec a \cdot (\vec b \cross \vec c) = 0$
\end{lemma}

\section{Lines and Planes}

\begin{definition}[Line]
	We define a line with a direction vector $\vec v = \langle a, b, c \rangle$

	\begin{equation}
		\vec r = \vec v_0 + t \vec v
	\end{equation}

	\textbf{Parametric Form}

	\begin{equation}
		\begin{cases}
			x &= x_0 + at\\
			y &= y_0 + bt\\
			z &= z_0 + ct
		\end{cases}
	\end{equation}

	\textbf{Symmetric Form}

	\begin{equation}
		t = \frac{x - x_0}{a} = \frac{y - y_0}{b} = \frac{z - z_0}{c}
	\end{equation}

	Notice how the symmetric form does not require parameters, it tells the relationship between the coordinates.
\end{definition}

\begin{example}
	Intersection problem
\end{example}

\begin{definition}[Plane]
	Given a point $P_0 \equiv \vec r_0$ and another point $P \equiv \vec r$ on the plane, along with the normal vector $\hat n = \langle a, b, c \rangle$.

	Now, we see that $\vec r - \vec r_0$ is always on the plane, so that it follows that

	\begin{align}
		\hat n \cdot (\vec r - \vec r_0) &= 0\\
		a(x - x_0) + b(y - y_0) + c(z - z_0) &= 0\\
		ax + by + cx &= d
	\end{align}

	where $d = ax_0 + by_0 + cz_0$
\end{definition}

\begin{example}
	Given $A(2, 0, 3), B(0, -4, 6), C(-3, 6, 0)$,  on a plane, find the equation of the plane.
\end{example}

\begin{sol}
	We find that

	\begin{equation}
		\overrightarrow{AB} \cross \overrightarrow{AC} = -8 \langle 2, 3, 4 \rangle
	\end{equation}

	We take any point and compute $d$

	\begin{equation}
		d = 2\cdot 2 + 0 \cdot 3 + 3 \cdot 4 = 16
	\end{equation}
	
	so the equation is

	\begin{equation}
		2x + 3y + 4z = 16
	\end{equation}
	
	\textit{What if we want to sketch the plane?}

	We simply find the $x,y,z$-intersection of the plane, label them on a skeleton, then connect they for a triangle.
\end{sol}

\begin{example}
	Given two planes

	\begin{equation}
		\begin{cases}
			x+y+z &= 1\\
			x-2y+3z &= 1
		\end{cases}
	\end{equation}

	\begin{enumerate}[a)]
		\item Find the angle between the two planes
		\item Find the equation of the intersecting line
	\end{enumerate}
\end{example}

\begin{sol}
	\begin{enumerate}[a)]
		\item We have the normal vectors

		\begin{equation}
			\begin{cases}
				\hat n_1 &= \langle 1, 1, 1 \rangle\\
				\hat n_2 &= \langle 1, -2, 3 \rangle
			\end{cases}
		\end{equation}

		We simply find the angle between them using the dot product.

		\begin{equation}
			\arccos\left(\frac{\vec a \cdot \vec b}{ab}\right) = \arccos\left(\frac{2}{\sqrt{42}}\right)
		\end{equation}

		\item We need the direction vector and a point on the line.
		
		We can find a point on the line by defining either $x, y,$ or $z$ for the two equations and solve for the other variables. (e.g. A point here on the line is $P(1,0,0)$)

		For the direction vector, we can cross the normal vectors $\vec{n_1} \cross \vec{n_2}$ to find the vector.
	\end{enumerate}
\end{sol}

\begin{definition}[Distance Between a Point and a Plane]
	Given some point $P$ and a random point $A$ on the plane, we can have some vector $\overrightarrow{AP}$, which, if we project onto the normal vector $\hat n$ of the plane, will give us the component of the vector $\overrightarrow{AP}$ parallel to the normal vector.

	\begin{equation}
		d = \abs{\overrightarrow{AP}} \abs{\cos\theta} = \abs{\overrightarrow{AP} \cdot \hat n}
	\end{equation}

	where $\theta$ is the angle between the $\overrightarrow{AP}$ and $\hat n$
\end{definition}

\begin{example}
	We want to find the distance between two paralle planes.
\end{example}

\begin{sol}
	Simply find a vector that ``connects'' the two planes, let that vector be $\vec v$. Then, calculate $\abs{\vec v \cdot \hat n}$ where $\hat n$ is the normal vector of the plane.
\end{sol}