\documentclass[oneside, 11pt]{book}
\usepackage{xeCJK} % Support for CJK texts
\usepackage[letterpaper]{geometry} % Geometry (or a4paper)
\usepackage{multicol} % Flexibility in columns
% \usepackage[subtle]{savetrees} % For saving trees, perhaps
\usepackage[shortlabels]{enumitem} % Better enumerate
\usepackage[hidelinks]{hyperref} % Hyperref
\usepackage[parfill]{parskip} % Skip ahead, not indent
\usepackage{amsmath, amssymb, amsthm} % Math, symbols, and theorems
\usepackage{physics}
\usepackage{esint} % Extended set of integrals
\usepackage{commath} % Makes typing certain math expressions much easier!
\usepackage{siunitx} % Unit typesetting
\usepackage{graphicx} % Better graphics
\usepackage{cleveref} % Better references
\usepackage{booktabs} % Better tables
\usepackage{tikz} % Figures!
\usetikzlibrary{calc,positioning}
% \usepackage{biblatex}

\usepackage{caption}
\usepackage{subcaption}
\usepackage{float}
\usepackage{blindtext}
\usepackage[most, breakable]{tcolorbox}
\usepackage{nicefrac}
\usepackage{changepage}
\usepackage{bm}
\usepackage{stmaryrd}

% cleveref setup
\crefname{figure}{Figure}{Figure}
\crefname{table}{Table}{Table}
\crefname{equation}{Eq.}{Eq.}

% geometry
\geometry{margin=1in}

% siunitx setup
\sisetup{
    per-mode=fraction,
    fraction-function=\tfrac
}

% hyperref setup
\hypersetup{
    colorlinks=true,
    linkcolor=blue,
    urlcolor=blue
}

% change QED square to black square
\renewcommand{\qedsymbol}{$\blacksquare$}

% reset enumeration at the start of align
%\AtBeginEnvironment{align}{\setcounter{equation}{0}}

% change emptyset symbol
\renewcommand*{\emptyset}{\varnothing}

% make differentials easier to type
\newcommand*{\D}{\,\differential}

% range and span
\newcommand*{\spn}{\ensuremath{\mathrm{span}}}
\newcommand*{\rng}{\ensuremath{\mathrm{rng}}}
\newcommand*{\diag}{\ensuremath{\mathrm{diag}}}

% vector bold text
\newcommand{\bv}[1]{\bm{\mathrm{#1}}}

% add environments for different theorems and definitions
\newtheoremstyle{break}
  {\topsep}{\topsep}%
  {\normalfont}{}%
  {\bfseries}{}%
  {\newline}{}%

\newtheoremstyle{other}
  {\topsep}{\topsep}%
  {\normalfont}{}%
  {\bfseries}{}%
  {\newline}{}%

\theoremstyle{break}
\newtheorem{theo}{Theorem}[section]

\theoremstyle{other}
\newtheorem{corollary}[theo]{Corollary}
\newtheorem{lemma}[theo]{Lemma}
\newtheorem{defi}{Definition}[section]
\newtheorem*{remark}{Remark}
\newtheorem{examp}{Example}[section]

\newtcolorbox{theorembox}{
	breakable,
	colframe=blue
}

\newtcolorbox{defbox}{
	breakable,
	colback=pink,
	boxrule=0pt
}

\newenvironment{theorem}
  {\begin{theorembox}\begin{theo}}
  {\end{theo}\end{theorembox}}

\newenvironment{definition}
	{\begin{defbox}\begin{defi}}
	{\end{defi}\end{defbox}}

\newenvironment{example}
	{\begin{tcolorbox}[breakable]\begin{examp}}
	{\end{examp}\end{tcolorbox}}

\newenvironment{indt}
	{\begin{adjustwidth}{3em}{0em}}
	{\end{adjustwidth}}

\newenvironment{sol}
	{\bigskip\textbf{\textit{Solution:}}\begin{indt}}{\end{indt}}


\title{MATH 229: Calculus III for Engineers\\
	\large Takahiro Sakai}
\author{Martin Gong / 七海喬介} 
\date{Jan 8 -- ??, 2024}

\begin{document}
	\maketitle

	\frontmatter

	\tableofcontents % Foreword ⬇

	\mainmatter % Chapters ⬇

	\chapter{Vector and the Geometry of Space}

\section{3-Dimensional Space}

\subsection{2D Coordinates}

\begin{equation}
	\mathbb{R}^2 = \left\{(x,y) \mid x,y \in \mathbb{R}\right\}
\end{equation}

\subsection{3D Coordinates}

\begin{equation}
	\mathbb{R}^3 = \left\{(x,y,z) \mid x,y,z \in \mathbb{R}\right\}
\end{equation}

\begin{lemma}[Distance Between 2 Points]
	\begin{equation}
		\abs{P_1 P_2} = \sqrt{(x_1 - x_2)^2 + (y_1 - y_2)^2 + (z_1 - z_2)^2}
	\end{equation}
\end{lemma}

\begin{proof}
	Easily proven by using the Pythagorean Theorem twice.
\end{proof}

\begin{lemma}[Spherical Surface]
	Given point $C(a,b,c)$ and $P(x,y,z)$ where $P$ is a point on the spherical surface and $r$ is the radius of the sphere.
	
	\begin{equation}
		(x - a)^2 + (y - b)^2 + (z - c)^2 = r^2
	\end{equation}

	To define a solid spherical space

	\begin{equation}
		\sqrt{(x - a)^2 + (y - b)^2 + (z - c)^2} \leq r
	\end{equation}
\end{lemma}

\section{Vectors}

\begin{definition}[Vector]
	Vector is a quantity that has a \textbf{magnitude} and a \textbf{direction}.
\end{definition}

We say that two vectors $\vec{u}$ and $\vec{v}$ are equal if they have the same length and direction.

\subsection{Vector Operation}

Omitted

\subsection{Components}

In $\mathbb{R}^2$

\begin{equation}
	\vec{a} \equiv \langle a_1, a_2 \rangle
\end{equation}

In $\mathbb{R}^3$

\begin{equation}
	\begin{cases}
		\vec{a} &\equiv \langle a_1, a_2, a_3 \rangle\\
		\vec{0} &\equiv \langle 0, 0, 0, \rangle
	\end{cases}
\end{equation}

\begin{definition}
	Length of $\vec{a} \equiv \langle a_1, a_2, a_3 \rangle$ is

	\begin{equation}
		\abs{\vec{a}} = \sqrt{{a_1}^2 + {a_2}^2 + {a_3}^2}
	\end{equation}
\end{definition}

\subsection{Standard Basis Vectors}

\begin{equation}
	\begin{cases}
		\hat{i} &= \langle 1, 0, 0 \rangle\\
		\hat{j} &= \langle 0, 1, 0 \rangle\\
		\hat{k} &= \langle 0, 0, 1 \rangle\\
	\end{cases}
\end{equation}

\section{The Dot Products}

\begin{definition}
	\begin{equation}
		\vec a = \langle a_1, a_2, a_3 \rangle \qquad \vec b = \langle b_1, b_2, b_3 \rangle
	\end{equation}

	Then, the dot product is

	\begin{equation}
		\vec a \cdot \vec b \equiv a_1 b_1 + a_2 b_2 + a_3 b_3 
	\end{equation}
\end{definition}

\textbf{Properties}

\begin{enumerate}[1.]
	\item $\vec a \cdot \vec a = {a_1}^2 + {a_2}^2 + {a_3}^2 = \abs{\vec a}^2$
	\item $\vec a \cdot \vec b = \vec b \cdot \vec a$
	\item $\vec a \cdot (\vec b + \vec c) = \vec a \cdot \vec b + \vec a \cdot \vec c$
	\item $(c\vec a) \cdot \vec b = c(\vec a \cdot \vec b)$
	\item $\vec 0 \cdot \vec a = 0$
\end{enumerate}

\begin{theorem}
	\begin{align}
		\vec a \cdot \vec b &= \abs{\vec a} \abs{\vec b} \cos\theta\\
		\cos\theta &= \frac{\vec a \cdot \vec b}{\abs{\vec a} \abs{\vec b}}, 0 \leq \theta \leq \pi
	\end{align}
\end{theorem}

\begin{lemma}
	\begin{itemize}
		\item If $\vec a \cdot \vec b > 0$ then $\cos \theta > 0 \implies \theta < \frac{\pi}{2}$
		\item If $\vec a \cdot \vec b < 0$ then $\cos \theta < 0 \implies \theta > \frac{\pi}{2}$
		\item If $\vec a \cdot \vec b = 0$, then $\theta = \frac{\pi}{2}, \vec a \perp \vec b$
	\end{itemize}
\end{lemma}

\subsection{Law of Cosine}

\begin{equation}
	\abs{\vec a - \vec b}^2 = \abs{\vec a}^2 + \abs{\vec b}^2 - 2 \abs{\vec a}\abs{\vec b}\cos\theta
\end{equation}

\begin{proof}
	\begin{align}
		\abs{\vec a - \vec b}^2 &= (\vec a - \vec b) \cdot (\vec a - \vec b)\\
		&= \abs{\vec a}^2 - 2 \vec a \cdot \vec b + \abs{\vec b}^2\\
		&= \abs{\vec a}^2 + \abs{\vec b}^2 - 2ab\cos(\theta)
	\end{align}	
\end{proof}

\subsection{Projection}
\begin{figure}[H]
	\centering
		\begin{tikzpicture}[dot/.style={circle,inner sep=1pt,fill,label={#1},name=#1},
			extended line/.style={shorten >=-#1,shorten <=-#1},
			extended line/.default=1cm]
			\draw[thick,-stealth] (-4.5,0) -- (4.5,0);
			\draw[thick,-stealth] (0,0) -- (0,4.5);
			\coordinate (A) at (0,0);
			\coordinate (B) at (-4,3);
			\draw [extended line=0.5cm, stealth-stealth] (A) -- (B) node[pos=1.15,font=\small]{$\vec a$};     
			\draw [ -stealth] (0,0) -- (-2.6, 4.3) coordinate (yn) node[right]{$\,\vec b$}; 
			\draw[dashed] (yn) --  node[midway,above left]{$\varepsilon$} ($(A)!(yn)!(B)$) node[below left]{$\vec b_p$};
		\end{tikzpicture}  
	\caption{Projection}
	\label{fig:projection}
\end{figure}

\textbf{Add to this.}

\begin{align}
	\abs{\vec b}
\end{align}

\begin{example}
	\begin{equation}
		\vec u = \langle 1,1,2\rangle \qquad \vec v = \langle -2,3,1\rangle
	\end{equation}

	Find projection of $\vec u$ onto $\vec v$
\end{example}

\begin{sol}
	\begin{align}
		\mathrm{comp}_{\vec c} \vec u &= \vec u \cdot \frac{\vec v}{\abs{\vec v}}\\
		&= \frac{-2 + 3 + 2}{\sqrt{14}} = \frac{3}{\sqrt{14}}
	\end{align}

	\begin{equation}
		\mathrm{proj}_{\vec v} \vec u = \left(\mathrm{comp}_{\vec v} \vec u\right) \frac{\vec v}{\abs{\vec v}} = \frac{3}{\sqrt{14}} \cdot \frac{\vec v}{\sqrt{v}} = \frac{3}{14} \vec v
	\end{equation}
\end{sol}

\subsection{Work}

Move an an object from $P$ to $Q$ with a force $\vec F$ forming an angle $\theta$ with the displacement vector $\vec D$.

\begin{align}
	\text{Work} &\equiv \text{Force} \cross \text{Dist}\\
	W &= (\abs{\vec F}\cos \theta) \abs{\vec D}\\
	&= \abs{\vec F}\abs{\vec D}\cos \theta\\
	&= \vec F \cdot \vec D\\
	&\implies W = \vec F \cdot \vec D
\end{align}

\begin{example}
	Move a particle from $P(2,1,0)[m]$ to $Q(4,6,2)$ with a force $\vec F = \langle 3, 4, 5 \langle [N]$.

	What is the work done by $\vec F$?
\end{example}

\begin{sol}
	\begin{align}
		W &= \vec F \cdot \vec{PQ}\\
		&= \langle 3, 4, 5 \rangle \cdot \langle 2, 5, 2\rangle\\
		&= \SI{36}{\N\m}
	\end{align}
\end{sol}

\section{The Cross Product}

\begin{definition}
	Given the vectors

	\begin{equation}
		\vec a = \langle a_1, a_2, a_3 \rangle, \vec b = \langle b_1, b_2, b_3 \rangle
	\end{equation}

	The cross product is defined as

	\begin{equation}
		\vec a \cross \vec b = \begin{vmatrix}
			\hat i & \hat j & \hat k\\
			a_1 & a_2 & a_3\\
			b_1 & b_2 & b_3
		\end{vmatrix} = \langle a_2 b_3 - a_3 b_2, a_3 b_1 - a_1 b_3, a_1 b_2 - a_2 b_1 \rangle
	\end{equation}
\end{definition}

\textbf{Properties of the Dot Product}

\begin{enumerate}[1.]
	\item $(\vec a \cross \vec b) \perp \vec a \& \vec b$ and the direction follows the right-hand rule.
	\item $\abs{\vec a \cross \vec b} = \abs{\vec a} \abs{\vec b} \sin\theta, 0 \leq \theta \leq \pi$
	\item $\abs{\vec a \cross \vec b} = $ the area of the parallelogram formed by the two vectors.
	\item If $\vec a \parallel \vec b$, then $\vec a \cross \vec b = \vec 0$
	\item Cross product of basis vectors
	
	\begin{equation}
		\begin{cases}
			\hat i \cross \hat j &= \hat k\\
			\hat j \cross \hat k &= \hat i\\
			\hat k \cross \hat i &= \hat j
		\end{cases}
	\end{equation}

	\item The cross product is not commutative
	\item The cross product is not associative

	\begin{example}
		\begin{equation}
			\begin{cases}
				\hat i \cross (\hat i \cross \hat j) &= \hat i \cross \hat k = -\hat j\\
				(\hat i \cross \hat i) \cross \hat j &= \vec 0 \cross \hat j = \vec 0
			\end{cases}
		\end{equation}
	\end{example}

	\item You can find the normal vector to a plane by applying the cross product to two non-parallel vectors on that plane.
\end{enumerate}

\begin{example}
	Given points

	\begin{equation*}
		P(1,4,6), Q(-2,5,1), R(1,-1,1)
	\end{equation*}

	that lie on a plane

	\begin{enumerate}[a)]
		\item Find the vector normal to the plane
		\item Find the area of $\triangle{PQR}$
	\end{enumerate}
\end{example}

	\appendix

\end{document}