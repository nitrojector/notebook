\chapter{Linear Transformations}

\section{Definitions and Examples}

\begin{definition}
	Let $V$ and $W$ be vector spaces. A \textbf{linear transformation} is a map $T:V \to W$ that satisfies the following properties:

	\begin{enumerate}[1)]
		\item $\forall u, v \in V, T(u + v) = T(u) + T(v)$
		\item $\forall u \in V, \forall \lambda \in \mathbb{R}, T(\lambda u) = \lambda T(u)$
	\end{enumerate}
\end{definition}

\begin{remark}
	\begin{enumerate}[1)]
		\item The two conditions $\iff \forall u, v \in V, \forall \lambda \in \mathbb{R}, T(u + \lambda v) = T(u) + \lambda T(v)$
		\item \underline{More generally:} $\forall v_1, v_2, \ldots, v_p \in V, \forall \alpha_1, \ldots, \alpha_p \in \mathbb{R}$
		
		\begin{equation} \label{eq:map-def}
			T\left(\sum_{i=1}^p \alpha_i v_i\right) = \sum_{i=1}^p \alpha_i T(v_i)
		\end{equation}

		\item $T(0_v) = 0_w$
		
		$T(0_v) = T(0_v + 0_v) = T(0_v) + T(0_v) \implies T(0_v) = 0_w$
	\end{enumerate}
\end{remark}

\begin{example}
	Let $A$ be a $m \cross n$ matrix

	$T_A:\mathbb{R}^n \mapsto \mathbb{R}^n$ is a linear map: it is the linear map associated with the $m \cross n$ matrix $A$.

	Let $x, y \in \mathbb{R}^n, \lambda \in \mathbb{R}$

	Then,
	
	\begin{align}
		T_A(x+\lambda y) &= A(x + \lambda y)\\
		&= Ax + \lambda A y\\
		&= T_A(x) + \lambda T_A(y)
	\end{align}

	$T_A$ is linear by theorem.
\end{example}

\begin{theorem}
	Let

	\begin{equation} \label{eq:matrix-map-thm}
		T: \mathbb{R}^n \to \mathbb{R}^m
	\end{equation}

	be a linear map. There exists a (unique) $m\cross n$ matrix $A$ such that $T = T_A$
	
	\begin{proof}
		Let $T:\mathbb{R}^n \to \mathbb{R}^m$ be a linear map

		Let $x = \begin{bmatrix}
			x_1\\\vdots\\x_n
		\end{bmatrix} \in \mathbb{R}^n$

		\begin{align}
			T(n) &= T\left(\sum_{i=1}^{n}x_i e_i\right)\\
			&= \sum_{i=1}^{n} x_i T(e_i)\\
			&= \begin{bmatrix}
				T(e_0) \mid \cdots \mid T(e_n)
			\end{bmatrix} \begin{bmatrix}
				x_1\\x_2\\\vdots\\x_n
			\end{bmatrix}
		\end{align}

		% Thus, $\forall n \in \mathbb{R}^n, T(x) = Ax \implies T = T_$
	\end{proof}
\end{theorem}

\begin{example}
	Given
	
	\[T\left(\begin{bmatrix}
		x\\y\\z
	\end{bmatrix}\right)=\begin{bmatrix}
		2x-y\\x+y+2z
	\end{bmatrix}\]

	is a linear map $T: \mathbb{R}^3 \to \mathbb{R}^2$ 

	\[A = \begin{bmatrix}
		2&0&-1\\1&1&2
	\end{bmatrix}, A \begin{bmatrix}
		x\\y\\z
	\end{bmatrix} \implies T = T_A\]
\end{example}

\section{Kernel and Range of a Linear Map}

Let $T:V \to W$ be a linear map

\begin{definition}
	The \textbf{Kernel} of $T$ is $\ker(T) = \left\{v \in V \mid T(v) = 0\right\}$
\end{definition}

\begin{definition}
	The \textbf{Range} of $T$ is $\rng(T) = \left\{w \in W \mid \exists v \in V, w = T(v)\right\}$
\end{definition}

\begin{remark}
	\begin{enumerate}
		\item $\ker(T)$ is a subspsace of $V$
		
		\begin{proof}
			Let $u, v \in \mathrm{ker}(T)$, let $\lambda \in \mathbb{R}$

			\begin{align*}
				T(u + \lambda v) &= T(u) + \lambda T(v)\\
				&= 0
			\end{align*}

			We know that $T$ is linear,

			$V$ being a vector space, $\ker(T)$ is a subspace by the theorem.
		\end{proof}

		\item $\rng(T)$ is a subspace of $W$

		\begin{proof}
			Let $w_1, w_2 \in \rng(T)$, let $\lambda \in \mathbb{R}$

			\begin{align*}
				w_1 + \lambda w_2 &= T(v_1) + \lambda T(v_2)\\
				&= T(v_1 + \lambda v_2)
			\end{align*}

			Since given our assumption $v_1, v_2 \in V$ and $V$ is a vector space, by properties of a vector space, $v_1 + \lambda v_2 \in V$

			This means that $w_1 + \lambda w_2 \in \rng(T)$

			By theorem, $W$ being a vector space, $\rng(T)$ is a subspace of $W$
		\end{proof}

		\item If $B = \{v_i\}_{i \in [1,n]}$ is a basis of $V$, then $\rng(T) = \spn\{T(v_i)\}_{i \in [1,n]}$
		
		\begin{proof}
			Let $\{v_i\}_i$ be a basis of $V$

			Let $w \in \rng(T) \iff \exists v \in V$ such that $w \ T(v)$

			Let $\{\alpha_i\}_i \subseteq \mathbb{R}$ such that $v \in \sum_i \alpha_i v_i$ since $\{v_i\}_i$ basis of $V$.

			We have: $w = T(v) = T(\sum_i \alpha_i v_i) = \sum_i \alpha_i T(v_i) \in \spn\{T(v_i)\}_i$

			Therefore, $\rng(T) \subseteq \spn\{T(v_i)\}_i$

			We also know that $\rng(T) \supseteq \spn\{T(v_i)\}_i$ because we can go backwards in the equality above, and see that any vector of the span must be a $w \in T(v)$ 
		\end{proof}
	\end{enumerate}
\end{remark}

\begin{theorem}[Rank-Nullity]
	Let $T:V \to W$ a linear map, where $V,W$ are finite dimensional, $\dim(V) < +\infty$

	\[\dim(V) = \dim(\ker(T)) + \dim(\rng(T))\]
\end{theorem}

\begin{example}
	Determine a basis for $\ker(T)$ and $\rng(T)$ where $T:\mathbb{R}^3 \to \mathbb{R}^2$

	\[\begin{bmatrix}
		x\\y\\z
	\end{bmatrix} \mapsto \begin{bmatrix}
		2x - z\\x+y+2z
	\end{bmatrix}\]
\end{example}

\begin{sol}
	Let $\begin{bmatrix}
		x\\y\\z
	\end{bmatrix} \in \ker(T)$ then $T\left(\begin{bmatrix}
		x\\y\\z
	\end{bmatrix}\right) = \begin{bmatrix}
		0\\0
	\end{bmatrix}$

	Row reduce and find that $\ker(T) = \spn\left\{\begin{bmatrix}
		1\\-5\\2
	\end{bmatrix}\right\}$

	\textit{then find the range}

	$\dim(\ker(T)) = 1$

	$3 = 1 + \dim(\rng(T)) \implies \dim(\rng(T)) = 2$
\end{sol}

\begin{example}
	Let $T: P_3(\mathbb{R}) \to P_3(\mathbb{R})$

	$p(x) \mapsto p(x) - (x+1) p'(x)$

	\begin{enumerate}
		\item Show that $T$ is linear
		\item Find a basis of $\ker(T)$
		\item Find a basis of $\rng(T)$
	\end{enumerate}
\end{example}

\begin{sol}
	\begin{enumerate}
		\item Let $p, q \in P_3(\mathbb{R})$, let $\lambda \in \mathbb{R}$
		
		\begin{align*}
			T(p + \lambda q) &= (p+\lambda q)(x) - (x+1)(p + \lambda q)'(x)\\
			&= \left[p(x) - (x+1)p'(x)\right] + \lambda \left[q(x) - (x+1)q'(x)\right]
		\end{align*}

		Thus, $T$ is linear.

		\item Let $p \in \ker(T)$, $p(x) = ax^3 + bx^2 + cx + d$
		
		$T(p) = 0 \iff \forall x, p(x) = (x+1) p'(x) = 0$

		We can expand it to the form

		\begin{align*}
			p(x) - (x+1)p'(x) &= 0\\
			ax^3 + bx^2 + cx + d + (x+1)(3ax^2 + 2bx + c) &= 0\\
			(-2a)x^3 + (-b-3a)x^2 + (-2b)x + (d-c) &= 0
		\end{align*}

		which means

		$\begin{cases}
			-2a = 0 &\implies a = 0\\
			-b-3a = 0\\
			-2b = 0 &\implies b = 0\\
			d-c = 0 &\implies c = d
		\end{cases}$

		\begin{tcolorbox}
			We know that if two polynomials are equivalent, their coefficients are the same because $\{1,x,x^2,x^3\}$ is linearly independent.
		\end{tcolorbox}

		$p(x) = dx + d = d(x+1), d \in \mathbb{R}$

		$\therefore \ker(T) = \spn\{(x+1)\}$

		\item $\dim(\rng(T)) = \dim(P_3(\mathbb{R})) - \dim(\rng(T)) = 4 - 1 = 3$

		We have

		\begin{align*}
			\rng &= \spn\left\{T(1), T(x), T(x^2), T(x^3)\right\}\\
			&= \spn\left\{1, -1, -x^2 - 2x, -2x^3 - 3x^2\right\}\\
			&= \spn\left\{1, -x^2 - 2x, -2x^3 - 3x^2\right\}
		\end{align*}

		It is a spanning set of $\rng(T)$ and $\dim(\rng(T)) = 3$

		Thus, it is a basis.
	\end{enumerate}
\end{sol}

\section{Properties of Linear Maps}

\begin{definition}
	Let $T:V \to W$ be a linear map

	\begin{enumerate}
		\item $T$ is \textbf{one-to-one / injective} if it satisfies the following condition:
		
		\[\forall v_1, v_2, \text{if } T(v_1) = T(v_2), \text{then } v_1 = v_2\]

		\item $T$ is \textbf{onto / surjective} if $\rng(T) = W$

		\[\forall w \in W, \exists v \in V \text{ such that } T(v) = w\]

		\item $T$ is bijective if it is both injective and surjective.
		\item $\forall w \in W, \exists ! v \in V \text{ such that } T(v) = w$
	\end{enumerate}
\end{definition}

\begin{definition}
	Let $T:V \to W$ be a linear map that is bijective. (\textbf{isomorphism}) The inverse map $T^{-1}: W \to V$ is defined by:

	\[\forall (v, w) \in V \cross W, T^{-1}(W) = V \text{ if } T(v) = w\]

	\textit{Properties}

	\begin{enumerate}
		\item $\forall v \in V, T^{-1}\left(T(v)\right) = v$
		\item $\forall w \in W, T\left(T_{-1}(w)\right) = w$
	\end{enumerate}
\end{definition}

\begin{theorem}
	If $T:V \to W$ is an isomorphism, then so is $T^{-1}:W \to V$
\end{theorem}

\begin{proof}
	Let $w_1, w_2 \in W, \lambda \in \mathbb{R}$

	\begin{align*}
		T^{-1}(w_1 + \lambda w_2) &= T^{-1}(T(v_1) + \lambda T(v_2))\footnotemark\\
		&= T^{-1}(T(v_1 + \lambda v_2))\\
		&= v_1 + \lambda v_2\\
		&= T^{-1}(w_1) + \lambda T^{-1}(w_2)\\
		&\therefore T \text{ is linear}
	\end{align*}
	\footnotetext{Where uniquely, $T(v_1) = w_1, T(v_2) = w_2$}
\end{proof}

\begin{theorem}
	Let $T: V \to W$ be a linear map.

	\begin{enumerate}
		\item $T$ is injective iff $\ker(T) = \{0\}$
		\item If $\dim(W) < +\infty$, then $T$ is surjective iff $\dim(\rng(T)) = \dim(W)$
	\end{enumerate}
\end{theorem}

\begin{proof}
	\begin{enumerate}
		\item Assume $T$ is injective
		
		Let $v \in \ker(T) \implies T(v) = 0 = T(0)$. Then, by the definition of injective maps, $v = 0$, thus $\ker(T) = \{0\}$

		Assum $\ker(T) = \{0\}$

		Let $v_1, v_2 \in V$ such that 
		
		\[T(v_1) = T(v_2) \implies T(v_1) - T(v_2) = T(v_1 - v_2) = 0 \implies v_1 - v_2 \in \ker(T) = \{0\}\]

		Thus $v_1 = v_2$
	\end{enumerate}
\end{proof}

\begin{theorem}
	Let $T: V \to W$ be a linear map

	Let $B = \{v_1, \ldots, v_n\}$ be a basis of the domain $V$.

	\begin{enumerate}
		\item $T$ is injective \textit{iff} $\{T(v_1), \ldots, T(v_n)\}$ is a family of linearly independent vectors in the codomain $W$
		\item $T$ is surjective \textit{iff} $\{T(v_1), \ldots, T(v_n)\}$ is a spanning set of the codomain $V$.
		\item $T$ is an isomorphism \textit{iff} $\{T(v_1), \ldots, T(v_n)\}$ is a basis of the codomain $W$
	\end{enumerate}
\end{theorem}

\begin{corollary}
	Let $T:V\to W$ be a linear map, $V, W$ finite dimensional.

	\begin{enumerate}
		\item If $T$ is injective, then $\dim(V) \leq \dim(W)$
		\item If $T$ is surjective, then $\dim(V) \geq \dim(W)$
		\item If $T$ is bijective/isomorphism, then $\dim(V) = \dim(W)$
	\end{enumerate}
\end{corollary}

\begin{example}
	Let $T: \mathbb{R}^3 \to \mathbb{R}^4$

	\[\begin{bmatrix}
		x_1\\x_2\\x_3
	\end{bmatrix} \mapsto \begin{bmatrix}
		x_1 - x_2\\x_2 + 2x_1\\x_2+x_1-2x_3\\x_3 - x_1
	\end{bmatrix}\]

	Show is one-to-one
\end{example}

\begin{sol}
	To show that the map is injective, show that the kernel space has a dimension of 0.
\end{sol}


\begin{example}
	Let $T:\mathbb{R}^3 \to \mathbb{R}^2$

	\[\begin{bmatrix}
		x_1\\x_2\\x_3
	\end{bmatrix} \mapsto \begin{bmatrix}
		2x_1 - x_3\\x_1+x_2+2x_3
	\end{bmatrix}\]

	Show that is onto.
\end{example}

\begin{sol}
	To show that the it is surjective, find the dimension of the range space, which is defined by

	\[\dim(\rng(T)) = \dim(\mathbb{R}^3) - \dim(\ker(T))\]

	We find that the kernel spaces has dimension of 1, thus dimension of the range space is 2.

	\[\dim(\rng(T)) = 2 = \dim(\mathbb{R}^2)\]

	Thus we've shown it is onto.
\end{sol}

\begin{example}
	Isomorphism problem, check image \verb|img0.HEIC|
\end{example}

\section{Matrix of a Linear Map}

\begin{tcolorbox}
	Fill in later
\end{tcolorbox}

\begin{example}
	Let $T: P_2(\mathbb{R}) to \mathbb{R}^3$
	
	$p(x) \mapsto \begin{bmatrix}
		p(1)\\p'(1)\\p''(1)
	\end{bmatrix}$

	Show that $T$ is an isomorphism. Then find the equivalent expression of $T^{-1}\left(\begin{bmatrix}
		a\\b\\c
	\end{bmatrix}\right)$

	\textit{Solution}

	Let
	$\begin{cases}
		\mathcal{B} &= \{1,x,x^2\} \text{ basis of } P_2(\mathbb{R})\\
		\mathcal{C} &= \{e_1, e_2, e_3\} \text{ standard basis of } \mathbb{R}^3
	\end{cases}$

	\[[T]_\mathcal{B}^\mathcal{C} = \left[[T(1)]_\mathcal{C}, [T(x)]_\mathcal{C}, [T(x^2)]_\mathcal{C}\right] = \begin{bmatrix}
		1&1&1\\0&1&2\\0&0&2
	\end{bmatrix} = A\]

	By the theorem we just have to prove that the transformation matrix is invertible, then we have shown that the transformation $T$ is an isomorphism.

	\[\det(A) = 2 \neq 0 : A \text{ is invertible } \iff \text{ is isomorphic}\]

	Then, we find 

	\[[T]_\mathcal{C}^\mathcal{B} = ([T]_\mathcal{C}^\mathcal{B})^{-1} = \begin{bmatrix}
		1&-1&\nicefrac{1}{2}\\
		0&1&-1\\
		0&0&\nicefrac{1}{2}
	\end{bmatrix}\]

	\[T^{-1}\left(\begin{bmatrix}
		a\\b\\c
	\end{bmatrix}\right) = a T^{-1}(e_1) + b T^{-1}(e_2) + c T^{-1}(e_3)\]

	We should at least state once that 

	\[[T^{-1}(e_3)]_\mathcal{B} = \begin{bmatrix}
		\nicefrac{1}{2}\\-1\\\nicefrac{1}{2}
	\end{bmatrix}\]

	The answer is

	\[T^{-1}\left(\begin{bmatrix}
		a\\b\\c
	\end{bmatrix}\right) = \left(\nicefrac{1}{2}\right) x^2 + (b-c) x + (a - b + \nicefrac{c}{2})\]
\end{example}

\begin{example}
	\[\begin{cases}
		T_1: &P_1(\mathbb{R}) \to M_2(\mathbb{R})\\
		&ax + b \mapsto \begin{bmatrix}
			a-b&0\\-2b&3b-a
		\end{bmatrix}\\
		T_2: & M_2(\mathbb{R}) \to \mathbb{R}\\
		&A \mapsto \tr(A)
	\end{cases} \qquad \begin{cases}
		\mathcal{B} &= \{1, x^2\}\\
		\mathcal{C} &= \left\{\begin{bmatrix}
			1&0\\0&0
		\end{bmatrix}, \begin{bmatrix}
			0&1\\0&0
		\end{bmatrix}, \begin{bmatrix}
			0&0\\1&0
		\end{bmatrix}, \begin{bmatrix}
			0&0\\0&1
		\end{bmatrix}, \right\}\\
		\mathcal{D} &= \{\frac{1}{5}\}
	\end{cases}\]

	\begin{enumerate}
		\item Find $[T_2T_1]^\mathcal{D}_\mathcal{B}$
		\item Verify that $[T_2T_1]_\mathcal{B}^\mathcal{R} = [T_2]_\mathcal{C}^\mathcal{D}[T_1]_\mathcal{B}^\mathcal{C}$
	\end{enumerate}
\end{example}

\begin{sol}
	\begin{enumerate}
		\item $[T_2T_1]^\mathcal{D}_\mathcal{B} = \begin{bmatrix}
			10 & 0
		\end{bmatrix}$
		\item \[[T_1]_\mathcal{B}^\mathcal{C} = \left[\left[T_1(1)\right]_\mathcal{C} \mid \left[T_1(x)\right]_\mathcal{C}\right] = \left[\left[\begin{bmatrix}
			-1&0\\-2&3
		\end{bmatrix}\right]_\mathcal{C} \mid \left[T_1(x)\right]_\mathcal{C}\right] = \begin{bmatrix}
			-1&1\\0&0\\-2&0\\3&-1
		\end{bmatrix} = B\]

		Then, using the same technique, we find

		\[\left[T_2\right]^\mathcal{D}_\mathcal{C} = \begin{bmatrix}
			5&0&0&5
		\end{bmatrix} = C\]

		Then we find that 

		\[CB = \begin{bmatrix}
			10&0
		\end{bmatrix}\]
	\end{enumerate}
\end{sol}