\chapter{System of Coordinates}

\section{Coordinate of a Vector Relative to a Basis}

Let $B = \{v_1, v_2,\ldots,v_n\}$ be a basis of a vector space $V$.

Then $\forall v \in V, v = x_1 v_1 + x_2 v_2 + \cdots x_n v_n, x_i \in \mathbb{R}$

The set of scalars $x_1, x_2, \ldots, x_n$ are unique.

\begin{definition}
	$[v]_B = \begin{bmatrix}
		x_1\\x_2\\\vdots\\x_n
	\end{bmatrix}$ is the coordinate vector of $v$ relative to the basis $B$.
\end{definition}

\begin{example}
	$V = \mathbb{R}^2, B' = \left\{
		\begin{bmatrix}
			1\\1\\0
		\end{bmatrix}, \begin{bmatrix}
			1\\-1\\0
		\end{bmatrix}, \begin{bmatrix}
			0\\2\\1
		\end{bmatrix}
	\right\}$

	Let $v = \begin{bmatrix}
		1\\2\\3
	\end{bmatrix}$

	One verifies that 

	\[v = \left(-\frac{1}{4}\right)v_1 + \left(\frac{5}{4}\right)v_2 + \left(\frac{7}{4}\right)v_3\]

	\[[v]_{B'} = \begin{bmatrix}
		-\frac{1}{4}\\\frac{5}{4}\\\frac{7}{4}
	\end{bmatrix}\]
\end{example}

\begin{example}
	$V = P_2(\mathbb{R}), B' = \left\{1, x+1, (x+1)^2\right\}$

	Let $p(x) = x^2 + 2x - 1$

	Then, $[p(x)]_{B'} = \renewcommand{\arraystretch}{1.5}\begin{bmatrix}
		-2\\0\\1
	\end{bmatrix}$
\end{example}

\begin{example}
	$V = P_n(\mathbb{R}), B = \left\{1,x,x^2,\ldots,x^n\right\}$

	Let $p(x) = a_nx^n + a_{-1}x^{n-1}+\cdots + a_1 x + a_0$ be a vector in $P_n(\mathbb{R})$

	Then, $[p(x)]_{B'} = \begin{bmatrix}
		a_0\\a_1\\\vdots\\a_n
	\end{bmatrix}$
\end{example}

\begin{example}
	$V = \mathbb{R}^n, B = \{e_1, e_2, \ldots, e_n\}$ canonical basis of $\mathbb{R}^n$

	Let $x = \begin{bmatrix}
		x_1\\\vdots\\x_n
	\end{bmatrix}$ be a vector in $\mathbb{R}^n$

	Then $[x]_B = \begin{bmatrix}
		x_1\\\vdots\\x_n
	\end{bmatrix}$
\end{example}

\section{Change of Basis}

Let $B = \{v_1, \ldots, v_n\}$ and $\mathcal{C} = \{w_1, \ldots, w_n\}$ be two bases of a vector space $V$.

Then, $\forall v \in V$, $[v]_B$: coordinate vector of $V$ relative to $B$, $[v]_\mathcal{C}$: ~ $\mathcal{C}$

We wish to relate the quantities $[v]_B$ and $[v]_\mathcal{C}$

\begin{definition}
	Let $P_{\mathcal{C} \leftarrow B}$ be the $n\cross n$ matrix defined by

	\begin{equation} \label{eq:change-of-basis}
		P_{\mathcal{C} \leftarrow B} = \begin{bmatrix}
			\left[v_1\right]_{\mathcal{C}} \mid \left[v_2\right]_{\mathcal{C}} \mid \cdots\mid \left[v_n\right]_{\mathcal{C}}
		\end{bmatrix}
	\end{equation}

	$P_{\mathcal{C} \leftarrow B}$ is the \textit{change of basis matrix} from $B$ to $\mathcal{C}$; change of coordinate.
\end{definition}

\begin{theorem}
	\begin{equation} \label{eq:cob-theorem}
		\forall v \in V, [v]_\mathcal{C} = P_{\mathcal{C} \leftarrow B} \cdot [v]_B
	\end{equation}
\end{theorem}

\begin{example}
	$V = \mathbb{R}^n, B = \left\{
		\begin{bmatrix}
			1\\0\\0
		\end{bmatrix}, \begin{bmatrix}
			0\\1\\0
		\end{bmatrix}, \begin{bmatrix}
			0\\0\\1
		\end{bmatrix}\right\}, B' = \left\{
			\begin{bmatrix}
				1\\1\\0
			\end{bmatrix}, \begin{bmatrix}
				1\\-1\\1
			\end{bmatrix}, \begin{bmatrix}
				0\\2\\1
			\end{bmatrix}
		\right\}$

	\begin{enumerate}[i)]
		\item Find $P_{B' \leftarrow B}$
		\item If $x = \begin{bmatrix}
			1\\2\\3
		\end{bmatrix}$, find $[x]_{B'}$
	\end{enumerate}
\end{example}

\begin{sol}
	\begin{enumerate}[i)]
		\item $P_{B' \leftarrow B} = \begin{bmatrix}
			3/4,1/4,-1/2\\1/4,-1/4,1/2\\-1/4,1/4,1/2
		\end{bmatrix}$
		
		\item $[x]_{B' \leftarrow B} [x]_B = \begin{bmatrix}
			-1/4\\5/4\\7/4
		\end{bmatrix}$
	\end{enumerate}
\end{sol}

\begin{theorem}
	\begin{equation} \label{eq:cob-inv}
		P_{B \leftarrow \mathcal{C}} = \left(P_{B \leftarrow \mathcal{C}}\right)^{-1}
	\end{equation}
\end{theorem}

\begin{example}
	$V = P_2(\mathbb{R}), B = \left\{1,x,x^2\right\}, \mathcal{C} = \left\{1, x+1, (x+1)^2\right\}$

	Let $p(x) = ax^2 + bx + c$. Find $[p(x)]_\mathcal{C}$ Then write down $p(x)$ as a linear combination of the elements in $\mathcal{C}$
\end{example}

\begin{sol}
	$[p(x)]_B = \begin{bmatrix}
		c\\b\\a
	\end{bmatrix}$

	$P_{B \leftarrow \mathcal{C}} = \begin{bmatrix}
		1 & 1 & 1\\
		0 & 1 & 2\\
		0 & 0 & 1
	\end{bmatrix}$

	$P_{B \leftarrow \mathcal{C}} = \left(P_{\mathcal{C} \leftarrow B}\right)^{-1}$

	$p(x) = (c-b+a)(1) + (b-2a)(x+1) + a(x+1)^2$
\end{sol}

\begin{theorem}
	$P_{\mathcal{D} \leftarrow B} = P_{\mathcal{D} \leftarrow \mathcal{C}} \cdot P_{\mathcal{C} \leftarrow B}$
\end{theorem}