\documentclass[oneside, 11pt]{book}
\usepackage{xeCJK} % Support for CJK texts
\usepackage[letterpaper]{geometry} % Geometry (or a4paper)
\usepackage{multicol} % Flexibility in columns
% \usepackage[subtle]{savetrees} % For saving trees, perhaps
\usepackage[shortlabels]{enumitem} % Better enumerate
\usepackage[hidelinks]{hyperref} % Hyperref
\usepackage[parfill]{parskip} % Skip ahead, not indent
\usepackage{amsmath, amssymb, amsthm} % Math, symbols, and theorems
\usepackage{physics}
\usepackage{esint} % Extended set of integrals
\usepackage{commath} % Makes typing certain math expressions much easier!
\usepackage{siunitx} % Unit typesetting
\usepackage{graphicx} % Better graphics
\usepackage{cleveref} % Better references
\usepackage{booktabs} % Better tables
\usepackage{tikz} % Figures!
\usetikzlibrary{arrows.meta,decorations.markings,calc,angles,quotes}
% \usepackage{biblatex}

\usepackage{caption}
\usepackage{subcaption}
\usepackage{float}
\usepackage{blindtext}
\usepackage[most, breakable]{tcolorbox}
\usepackage{nicefrac}
\usepackage{changepage}
\usepackage{bm}
\usepackage{stmaryrd}
\usepackage{mathtools}

% cleveref setup
\crefname{figure}{Figure}{Figure}
\crefname{table}{Table}{Table}
\crefname{equation}{Eq.}{Eq.}

% geometry
\geometry{margin=1in}

% siunitx setup
\sisetup{
    per-mode=fraction,
    fraction-function=\tfrac
}

% hyperref setup
\hypersetup{
    colorlinks=true,
    linkcolor=blue,
    urlcolor=blue
}

% change QED square to black square
\renewcommand{\qedsymbol}{$\blacksquare$}

% reset enumeration at the start of align
%\AtBeginEnvironment{align}{\setcounter{equation}{0}}

% change emptyset symbol
\renewcommand*{\emptyset}{\varnothing}

% make differentials easier to type
\newcommand*{\D}{\,\differential}

% range and span
\newcommand*{\spn}{\ensuremath{\mathrm{span}}}
\newcommand*{\rng}{\ensuremath{\mathrm{rng}}}
\newcommand*{\diag}{\ensuremath{\mathrm{diag}}}

% vector bold text
\newcommand{\bv}[1]{\bm{\mathrm{#1}}}

% add environments for different theorems and definitions
\newtheoremstyle{break}
  {\topsep}{\topsep}%
  {\normalfont}{}%
  {\bfseries}{}%
  {\newline}{}%

\newtheoremstyle{other}
  {\topsep}{\topsep}%
  {\normalfont}{}%
  {\bfseries}{}%
  {\newline}{}%

\theoremstyle{break}
\newtheorem{theo}{Theorem}[section]

\theoremstyle{other}
\newtheorem{corollary}[theo]{Corollary}
\newtheorem{lemma}[theo]{Lemma}
\newtheorem{defi}{Definition}[section]
\newtheorem*{rmrk}{Remark}
\newtheorem{examp}{Example}[section]

\newtcolorbox{theorembox}{
	breakable,
	colframe=blue
}

\newtcolorbox{defbox}{
	breakable,
	colback=pink,
	boxrule=0pt
}

\newenvironment{remark}
	{\begin{rmrk}}
	{\end{rmrk}\vspace{10pt}}

\newenvironment{theorem}
	{\begin{theorembox}\begin{theo}}
	{\end{theo}\end{theorembox}}

\newenvironment{definition}
	{\begin{defbox}\begin{defi}}
	{\end{defi}\end{defbox}}

\newenvironment{example}
	{\begin{tcolorbox}[breakable]\begin{examp}}
	{\end{examp}\end{tcolorbox}}

\newenvironment{indt}
	{\begin{adjustwidth}{3em}{0em}}
	{\end{adjustwidth}}

\newenvironment{sol}
	{\bigskip\textbf{\textit{Solution:}}\begin{indt}}{\end{indt}}

\numberwithin{equation}{section}


\title{PHYS 161 Lecture Notes\\
	\large Prof. Scott MacDonald}
\author{Martin Gong / 七海喬介} 
\date{Jan 9 - ??, 2024}

\begin{document}
	\maketitle

	\frontmatter

	\tableofcontents % Foreword ⬇

	\mainmatter % Chapters ⬇

	\chapter{Vector and the Geometry of Space}

\section{3-Dimensional Space}

\subsection{2D Coordinates}

\begin{equation}
	\mathbb{R}^2 = \left\{(x,y) \mid x,y \in \mathbb{R}\right\}
\end{equation}

\subsection{3D Coordinates}

\begin{equation}
	\mathbb{R}^3 = \left\{(x,y,z) \mid x,y,z \in \mathbb{R}\right\}
\end{equation}

\begin{lemma}[Distance Between 2 Points]
	\begin{equation}
		\abs{P_1 P_2} = \sqrt{(x_1 - x_2)^2 + (y_1 - y_2)^2 + (z_1 - z_2)^2}
	\end{equation}
\end{lemma}

\begin{proof}
	Easily proven by using the Pythagorean Theorem twice.
\end{proof}

\begin{lemma}[Spherical Surface]
	Given point $C(a,b,c)$ and $P(x,y,z)$ where $P$ is a point on the spherical surface and $r$ is the radius of the sphere.
	
	\begin{equation}
		(x - a)^2 + (y - b)^2 + (z - c)^2 = r^2
	\end{equation}

	To define a solid spherical space

	\begin{equation}
		\sqrt{(x - a)^2 + (y - b)^2 + (z - c)^2} \leq r
	\end{equation}
\end{lemma}

\section{Vectors}

\begin{definition}[Vector]
	Vector is a quantity that has a \textbf{magnitude} and a \textbf{direction}.
\end{definition}

We say that two vectors $\vec{u}$ and $\vec{v}$ are equal if they have the same length and direction.

\subsection{Vector Operation}

Omitted

\subsection{Components}

In $\mathbb{R}^2$

\begin{equation}
	\vec{a} \equiv \langle a_1, a_2 \rangle
\end{equation}

In $\mathbb{R}^3$

\begin{equation}
	\begin{cases}
		\vec{a} &\equiv \langle a_1, a_2, a_3 \rangle\\
		\vec{0} &\equiv \langle 0, 0, 0, \rangle
	\end{cases}
\end{equation}

\begin{definition}
	Length of $\vec{a} \equiv \langle a_1, a_2, a_3 \rangle$ is

	\begin{equation}
		\abs{\vec{a}} = \sqrt{{a_1}^2 + {a_2}^2 + {a_3}^2}
	\end{equation}
\end{definition}

\subsection{Standard Basis Vectors}

\begin{equation}
	\begin{cases}
		\hat{i} &= \langle 1, 0, 0 \rangle\\
		\hat{j} &= \langle 0, 1, 0 \rangle\\
		\hat{k} &= \langle 0, 0, 1 \rangle\\
	\end{cases}
\end{equation}

\section{The Dot Products}

\begin{definition}
	\begin{equation}
		\vec a = \langle a_1, a_2, a_3 \rangle \qquad \vec b = \langle b_1, b_2, b_3 \rangle
	\end{equation}

	Then, the dot product is

	\begin{equation}
		\vec a \cdot \vec b \equiv a_1 b_1 + a_2 b_2 + a_3 b_3 
	\end{equation}
\end{definition}

\textbf{Properties}

\begin{enumerate}[1.]
	\item $\vec a \cdot \vec a = {a_1}^2 + {a_2}^2 + {a_3}^2 = \abs{\vec a}^2$
	\item $\vec a \cdot \vec b = \vec b \cdot \vec a$
	\item $\vec a \cdot (\vec b + \vec c) = \vec a \cdot \vec b + \vec a \cdot \vec c$
	\item $(c\vec a) \cdot \vec b = c(\vec a \cdot \vec b)$
	\item $\vec 0 \cdot \vec a = 0$
\end{enumerate}

\begin{theorem}
	\begin{align}
		\vec a \cdot \vec b &= \abs{\vec a} \abs{\vec b} \cos\theta\\
		\cos\theta &= \frac{\vec a \cdot \vec b}{\abs{\vec a} \abs{\vec b}}, 0 \leq \theta \leq \pi
	\end{align}
\end{theorem}

\begin{lemma}
	\begin{itemize}
		\item If $\vec a \cdot \vec b > 0$ then $\cos \theta > 0 \implies \theta < \frac{\pi}{2}$
		\item If $\vec a \cdot \vec b < 0$ then $\cos \theta < 0 \implies \theta > \frac{\pi}{2}$
		\item If $\vec a \cdot \vec b = 0$, then $\theta = \frac{\pi}{2}, \vec a \perp \vec b$
	\end{itemize}
\end{lemma}

\subsection{Law of Cosine}

\begin{equation}
	\abs{\vec a - \vec b}^2 = \abs{\vec a}^2 + \abs{\vec b}^2 - 2 \abs{\vec a}\abs{\vec b}\cos\theta
\end{equation}

\begin{proof}
	\begin{align}
		\abs{\vec a - \vec b}^2 &= (\vec a - \vec b) \cdot (\vec a - \vec b)\\
		&= \abs{\vec a}^2 - 2 \vec a \cdot \vec b + \abs{\vec b}^2\\
		&= \abs{\vec a}^2 + \abs{\vec b}^2 - 2ab\cos(\theta)
	\end{align}	
\end{proof}

\subsection{Projection}
\begin{figure}[H]
	\centering
		\begin{tikzpicture}[dot/.style={circle,inner sep=1pt,fill,label={#1},name=#1},
			extended line/.style={shorten >=-#1,shorten <=-#1},
			extended line/.default=1cm]
			\draw[thick,-stealth] (-4.5,0) -- (4.5,0);
			\draw[thick,-stealth] (0,0) -- (0,4.5);
			\coordinate (A) at (0,0);
			\coordinate (B) at (-4,3);
			\draw [extended line=0.5cm, stealth-stealth] (A) -- (B) node[pos=1.15,font=\small]{$\vec a$};     
			\draw [ -stealth] (0,0) -- (-2.6, 4.3) coordinate (yn) node[right]{$\,\vec b$}; 
			\draw[dashed] (yn) --  node[midway,above left]{$\varepsilon$} ($(A)!(yn)!(B)$) node[below left]{$\vec b_p$};
		\end{tikzpicture}  
	\caption{Projection}
	\label{fig:projection}
\end{figure}

\textbf{Add to this.}

\begin{align}
	\abs{\vec b}
\end{align}

\begin{example}
	\begin{equation}
		\vec u = \langle 1,1,2\rangle \qquad \vec v = \langle -2,3,1\rangle
	\end{equation}

	Find projection of $\vec u$ onto $\vec v$
\end{example}

\begin{sol}
	\begin{align}
		\mathrm{comp}_{\vec c} \vec u &= \vec u \cdot \frac{\vec v}{\abs{\vec v}}\\
		&= \frac{-2 + 3 + 2}{\sqrt{14}} = \frac{3}{\sqrt{14}}
	\end{align}

	\begin{equation}
		\mathrm{proj}_{\vec v} \vec u = \left(\mathrm{comp}_{\vec v} \vec u\right) \frac{\vec v}{\abs{\vec v}} = \frac{3}{\sqrt{14}} \cdot \frac{\vec v}{\sqrt{v}} = \frac{3}{14} \vec v
	\end{equation}
\end{sol}

\subsection{Work}

Move an an object from $P$ to $Q$ with a force $\vec F$ forming an angle $\theta$ with the displacement vector $\vec D$.

\begin{align}
	\text{Work} &\equiv \text{Force} \cross \text{Dist}\\
	W &= (\abs{\vec F}\cos \theta) \abs{\vec D}\\
	&= \abs{\vec F}\abs{\vec D}\cos \theta\\
	&= \vec F \cdot \vec D\\
	&\implies W = \vec F \cdot \vec D
\end{align}

\begin{example}
	Move a particle from $P(2,1,0)[m]$ to $Q(4,6,2)$ with a force $\vec F = \langle 3, 4, 5 \langle [N]$.

	What is the work done by $\vec F$?
\end{example}

\begin{sol}
	\begin{align}
		W &= \vec F \cdot \vec{PQ}\\
		&= \langle 3, 4, 5 \rangle \cdot \langle 2, 5, 2\rangle\\
		&= \SI{36}{\N\m}
	\end{align}
\end{sol}

\section{The Cross Product}

\begin{definition}
	Given the vectors

	\begin{equation}
		\vec a = \langle a_1, a_2, a_3 \rangle, \vec b = \langle b_1, b_2, b_3 \rangle
	\end{equation}

	The cross product is defined as

	\begin{equation}
		\vec a \cross \vec b = \begin{vmatrix}
			\hat i & \hat j & \hat k\\
			a_1 & a_2 & a_3\\
			b_1 & b_2 & b_3
		\end{vmatrix} = \langle a_2 b_3 - a_3 b_2, a_3 b_1 - a_1 b_3, a_1 b_2 - a_2 b_1 \rangle
	\end{equation}
\end{definition}

\textbf{Properties of the Dot Product}

\begin{enumerate}[1.]
	\item $(\vec a \cross \vec b) \perp \vec a \& \vec b$ and the direction follows the right-hand rule.
	\item $\abs{\vec a \cross \vec b} = \abs{\vec a} \abs{\vec b} \sin\theta, 0 \leq \theta \leq \pi$
	\item $\abs{\vec a \cross \vec b} = $ the area of the parallelogram formed by the two vectors.
	\item If $\vec a \parallel \vec b$, then $\vec a \cross \vec b = \vec 0$
	\item Cross product of basis vectors
	
	\begin{equation}
		\begin{cases}
			\hat i \cross \hat j &= \hat k\\
			\hat j \cross \hat k &= \hat i\\
			\hat k \cross \hat i &= \hat j
		\end{cases}
	\end{equation}

	\item The cross product is not commutative
	\item The cross product is not associative

	\begin{example}
		\begin{equation}
			\begin{cases}
				\hat i \cross (\hat i \cross \hat j) &= \hat i \cross \hat k = -\hat j\\
				(\hat i \cross \hat i) \cross \hat j &= \vec 0 \cross \hat j = \vec 0
			\end{cases}
		\end{equation}
	\end{example}

	\item You can find the normal vector to a plane by applying the cross product to two non-parallel vectors on that plane.
\end{enumerate}

\begin{example}
	Given points

	\begin{equation*}
		P(1,4,6), Q(-2,5,1), R(1,-1,1)
	\end{equation*}

	that lie on a plane

	\begin{enumerate}[a)]
		\item Find the vector normal to the plane
		\item Find the area of $\triangle{PQR}$
	\end{enumerate}
\end{example}

	\chapter{Kinematics}

We have our position vector

\begin{equation}
	\vec r(t) = (x(t), y(t))
\end{equation}

We use $\vec r$ because it seems natural, it is the direction we are pointing in.

\begin{remark}
	Sometimes when reference to radial $\vec r$ is misleading, we use $\vec x(t)$.
\end{remark}

The change of the vector in space across time sweeps over some \textbf{trajectory}.

\section{Displacement}

\begin{definition}[Displacement]
	The \textit{displacement vector} $\Delta\vec r$ is a measure of where the particle went (which depends on the origin!).

	\begin{equation}
		\Delta\vec r \equiv \vec r_f - \vec r_i = \vec r(t_f) - \vec r(t_i)
	\end{equation}
\end{definition}

\begin{enumerate}
	\item $\norm{\Delta \vec r} \neq$ distance travelled in general
	\begin{itemize}
		\item distance traveled = arc length of trajectory
	\end{itemize}

	\item $\Delta \vec r$ is coordinate independent.
	
	Take two coordinate systems $S$ and $S'$. Let them be defined with the relation $\vec r = \vec r' + \vec R$ where $\vec r$ and $\vec r'$ are vectors in the respective coordinate systems.

	\begin{equation}
		\begin{cases}
			S: &\Delta \vec r = \vec r_f - \vec r_i\\
			S':&\Delta \vec r' = \vec {r_f}' - \vec {r_i}'
		\end{cases}
	\end{equation}

	If we plug in the relation, we realize that they are the same, $\Delta \vec r = \Delta \vec r'$
\end{enumerate}

\section{Velocity}

\begin{definition}[Average Velocity]
	\begin{equation}
		\vec v_{\mathrm{avg}} \equiv \frac{\Delta \vec r}{\Delta t}
	\end{equation}
\end{definition}

Let $\D \vec r$ be the infinitesimal displacement.

When we consider a smaller interval:

\begin{equation}
	\lim_{\Delta t \to 0} \implies \norm{\D \vec r} = \D r \quad \text{(distance traveled)}
\end{equation}

A small change to $t$ results in a small change in $\D S$ (the distance / speed), proportionally

\begin{align}
	&\D S \propto \D t\\
	\implies &\D S = \left(\frac{\D S}{\D t}\right) \D t
\end{align}

\begin{definition}[Velocity]
	AKA the \textit{instantaneous velocity}

	\begin{equation}
		\vec v (t) \equiv \frac{\D \vec r}{\D t}
	\end{equation}

	\begin{itemize}
		\item $\norm{\vec v}$ = speed
		\item $\hat v$ = direction of motion
	\end{itemize}
\end{definition}

\begin{remark}
	A note on average velocity:

	\begin{equation}
		\vec v_{\mathrm{avg}} = \frac{1}{\Delta t} \int_{t_i}^{t_f} \vec v(t) \D t = \frac{1}{\Delta t} \int_{t_i}^{t_f} \frac{\D \vec r}{\D t} \D t = \frac{\Delta \vec r}{\Delta t}
	\end{equation}

	Note also if we find the magnitude, it would not be the same as the average speed since the norm would go over the integrals instead of what is being integrated.
\end{remark}

\begin{itemize}
	\item $\vec v$ a vector, so write $\vec v(t) = \dot x \hat x + \dot y \hat y = \dot{\vec{r}}$
	\item Compare to frames of reference, $S$ \& $S'$
	
	Suppose $\dot{\vec{R}} \neq 0$.

	Then we have

	\begin{equation}
		\begin{cases}
			\vec r &= \vec r' + \vec R\\
			\vec v &= \vec v' + \vec V
		\end{cases}
	\end{equation}

	This is known as the Galilean transformations, which, at higher velocities, ``translates'' to the Lorentz transformations.
\end{itemize}

We can also obtain $\vec r(t)$ given $\vec v(t)$

\begin{equation}
	\Delta \vec r = \int \D \vec r = \int_{t_i}^{t_f} \vec v \D t
\end{equation}

and

\begin{align}
	\vec r(t) &= \vec r_i + \vec v_i (t - t_i)
\end{align}

	\chapter{Newton's Laws}

\section{Dynamics}

Newton's Laws provide the framework for the dynamics of classical particle motion.

Question of Classical Mechanics:

\begin{center}
	\textit{Given $\vec r_0$ and $\vec v_0$ of the particle, with mass $m$, determine its subsequent motion, $\vec r(t)$, for all time $t$.}
\end{center}

\subsection{Within Context}

Newton originally formulated the laws to solve the question of gravity -- along the way, he formulated concepts like forces and momentum.

\section{Newton's Laws}

\begin{definition}
	The three laws of motion:

	\begin{description}
		\item[Law of Invertia] A particle remains at rest or moving with constant velocity unless influenced by a force.
		\item[$\mathbf{F = m a}$] The change in a particle's motion (i.e. its acceleration) is proportional to the force impressed, as vectors. 
		\item[Action / Reaction] Forces come in pairs: to every action by one particle on another, there is an equal and opposite force in return. 
	\end{description}
\end{definition}

\subsection{First Law}

There exist \textit{intertial frames of reference}, that is, a frame in which a \textit{free particle}\footnote{particle subject to absolutely no influences} has constant velocity.

\begin{remark}
	Essentially, a frame at rest and a frame with constant velocity are the same.
\end{remark}

Mathematically, this is expressed as

\begin{equation}
	\frac{\D^2 \vec r}{\D^2 t} = 0
\end{equation}

\subsection{Second Law}

Denote the force by $\vec F$

Two different particels subject to the same force (e.g. a spring). After the unfluence of the force (e.g. left spring), particle 1 has speed $v_1$ and particle 2 has speed $v_2$.

Consider the ratio

\begin{equation}
	\frac{v_1}{v_2} \equiv \frac{m_2}{m_1}
\end{equation}

where $m_i$ is an intrinsic property of the $i$-th particle we call its mass [unit: kg].

\textit{Assumption:} $m$ is independent of $\vec F$ and $\vec v$.

So we can write a relation:

\begin{equation}
	m_1 v_1 = m_2 v_2
\end{equation}

Assume we start from rest and apply some force for some duration, then we have

\begin{equation}
	m_1 \Delta v_1 = m_2 \Delta v_2 = F \Delta t
\end{equation}

And thus we have

\begin{equation} \label{eq:force-mv}
	F \Delta t = m \Delta v \implies F = \frac{m \Delta v}{\Delta t} = \frac{\Delta (mv)}{\Delta t} 
\end{equation}

\begin{definition}
	Define the (physical) \textbf{momentum} of a particle to be

	\begin{equation}
		\vec p = m \vec v
	\end{equation}
\end{definition}

As so we have with \cref*{eq:force-mv} the following

\begin{equation}
	\vec F = \frac{\Delta \vec p}{\Delta t}
\end{equation}

\begin{enumerate}
	\item As $\Delta t \to 0$, we have that
	
	\begin{equation}
		\vec F = \frac{\D \vec p}{\D t}
	\end{equation}

	\item Forces (empirical) obey the \textit{principle of superposition}.

	\begin{equation}
		\vec F_\mathrm{net} = \sum_i \vec F_i
	\end{equation}
\end{enumerate}

Altogether we have that

\begin{equation}
	\vec F_\mathrm{net} = \frac{\D \vec p}{\D t}
\end{equation}

If $m$ is constant, then we have

\begin{equation}
	\frac{\D \vec p}{\D t} = m \frac{\D \vec v}{\D t} = m \vec a \implies \boxed{\vec F_\mathrm{net} = m \vec a}
\end{equation}

mass is a measure of an object's inertia -- \textit{tendency to persist in its state of motion}.

\subsection{Third Law}

\begin{definition}
	A force is a directed influence between pairs of particles.
\end{definition}

If force of 1 on 2 is $\vec F_{12}$,

then force of 2 on 1 is $\vec F_{21} = - \vec F_{12}$.

\textbf{IMPORTANT:} Forces always come in pairs! (e.g. When we are sitting on our seats, its us pushing on the seat, and the seat pushing on us. The force of us pushing on the seat comes from gravity.)

\begin{example}
	Given $\vec r_0$, $\vec v_0$, and $m$, fint $\vec r(t)$

	Newton's laws:

	\begin{enumerate}
		\item go to an intertial frame: $\vec r(t)$
		\item Identify forces acting on particle: $\vec F$
		\item Then we just solve the differential equation.
		
		\begin{equation}
			\vec F_\mathrm{net} = m\frac{\D^2 \vec r}{\D t^2}
		\end{equation}

		Our initial conditions are the two givens.
	\end{enumerate}
\end{example}

\section{Scenarios of Newton's Laws}

\subsection{Constant Forces}

\subsection{Variable Forces with Time}

\subsection{Variable Forces with Position}

\subsection{Variable Forces with Velocity}

	\chapter{Energy}

\section{Derivation from Newton's Laws}

Work and energy can actually be thought as a consequence of Newton's laws and is another way of looking at the motion of objects. For more on the derivation, check provided lecture notes.

\begin{theorem}[Work-Energy Theorem]
	In simple form:
	\begin{equation}
		W = \Delta K
	\end{equation}

	In general form:
	\begin{equation}
		\int_C \vec F \cdot \D \vec l = \frac{1}{2}mv^2 - \frac{1}{2}m{v_0}^2
	\end{equation}
\end{theorem}

\section{Work \& Energy}

We know $\vec F \propto \vec a \implies \vec F \parallel \D \vec r$ changes speed; $\vec F \perp \D \vec r$ changes direction.

In our process of deriving $W = \Delta K$, we turned a vector equation into a scalar equation. Hence, here we are concerned only with $\vec F_\parallel$ (moreover, if I draw trajectory I know what $\vec F_\perp$ does but not $\vec F_\parallel$)

Evidently, $\vec F_\parallel \implies \vec F \cdot \D \vec r \propto$ change in speed.

\begin{equation}
	\D \vec v \cdot \vec v \sim \frac{1}{2}\D(v^2)
\end{equation}

More precisely, the 2nd law of motion:

\begin{equation}
	\vec F \cdot \D \vec r = \frac{1}{2} m \D(v^2)
\end{equation}

Infinitesimally, this would be:

\begin{equation}
	\vec F(\vec r_j) \cdot \D \vec r_j = \frac{1}{2} m \left[{v_{j+1}}^2 - {v_j}^2\right]
\end{equation}

We can informally sum all these parts as

\begin{equation}
	\sum_{\vec r = \vec r_0}^{\vec r} \vec F(\vec r_j) \cdot \D \vec r_j = \frac{1}{2} m \left(v^2 - v_0^2\right) = \int_{\vec r_0}^{\vec r} \vec F(\vec r) \cdot \D \vec r
\end{equation}

\textbf{Conclusion}

$W$ is adding up all the infinitesimal contributions of the force tangent to the trajectory (i.e. ones that change the speed) along the trajectory from start to finish.

$K$ measures the change in speed due to $\vec F$ as required by Newton's 2nd law.

$W$ is a sum of infinitesimal scalar quantities along a curve which is a line integral.

In cartesian:

\begin{equation}
	\int \vec F(\vec r) \cdot \D \vec r = \int \vec F(x, y, z) \cdot \left[\D x \hat x + \D y \hat y + \D z \hat z\right]
\end{equation}

\subsection{Kinetic Energy}

\begin{equation}
	K \equiv \frac{1}{2}mv^2
\end{equation}

\begin{itemize}
	\item Kinetic energy is, well, energy associated with motion.

	\item $K$ is frame-dependent.
	
	\item And the units are Joules.
\end{itemize}

\subsection{Work}

\begin{equation} \label{eq:work}
	W = \int_C \vec F \cdot \D \vec l
\end{equation}

Since $W = \Delta K$ is the change in energy of our particle/system.

\begin{itemize}
	\item If $W > 0 \implies K > K_0 \implies$ gained energy/speed
	\item If $W < 0 \implies K < K_0 \implies$ loses energy/speed
	\item If $W = 0 \implies K = K_0 \implies$ no change in energy/speed
\end{itemize}

Work is the energy transferred into/out-of a system by mechanical means (i.e. application of a fore over a displacement)

We say ``Work is done by force $\vec F$ along the path'' when writing \cref*{eq:work}.

It is useful to measure rate at which work is done -- called the \textbf{power} [units: Watts 1 W = 1 J/s]

\begin{align} 
	\frac{\D W}{\D t} &= \frac{\D}{\D t}(W)\\
	&= \frac{\D}{\D t} \left[\int_{\vec r_0}^{\vec r} \vec F(\vec r') \cdot \D \vec r'\right]\\
	&= \frac{\D}{\D t} \left[\int_{t_0}^t \vec F(\vec v) \cdot \vec v(t') \D t'\right]
\end{align}

And so we have:

\begin{equation} \label{eq:power}
	\frac{\D W}{\D t} = \vec F \cdot \vec V
\end{equation}

in this case, $\D W$ refers to an infinitesimal amount of work instead of it's change. It is technically $\mathrm{d}\hspace*{-0.08em}\bar{}\hspace*{0.1em} W$ -- inexact differential.

\begin{example}
	Consider constant force $\vec F = F_0 \hat n$, $F_0 = \mathrm{const}$, $\hat n$ constant unit vector.

	Compute work done over displacement $\Delta \vec r = \vec f - \vec f_0$
\end{example}

\begin{sol}
	\begin{equation}
		W = \vec F \cdot \Delta \vec r
	\end{equation}
\end{sol}

\begin{example}
	Consider a central force, $\vec F(\vec r) = f(r) \hat r$ work in 2-dim

	\begin{enumerate}[(a)]
		\item Show W is independent of path
		\item Let $f(r) = - A/r^2$ for $A > 0$ a constant. Find $v(r)$ if $v(r = r_0) = 0$.
	\end{enumerate}
\end{example}

\begin{sol}
	\begin{enumerate}[(a)]
		\item Work in polar coords
		
		$\D \vec l = \D r \hat r + r \D \theta \hat \theta$

		Thus, 

		\begin{align}
			W &= \int_C \vec F cdot \D \vec l\\
			&= \int_C f(r) \hat r \cdot (\D r \hat r + r \D \theta \hat \theta)\\
			&= \int_{r_0}^r f(r) \D r
		\end{align}

		Since it only requires the endpoints, it does not require the path, it is independent.

		\item $f(r) = -\frac{A}{r^2}$
		
		\begin{equation}
			W = \int_{r_0}^r -\frac{A}{r^2} \D r = A(\frac{1}{r} - \frac{1}{r_0})
		\end{equation}

		We can apply $W = \Delta K$

		\begin{align}
			A\left(\frac{1}{r} - \frac{1}{r_0}\right) &= \frac{1}{2}mv^2(r)\\
			\Aboxed{v(r) &= \pm \sqrt{\frac{2A}{m}\left(\frac{1}{r} - \frac{1}{r_0}\right)}}
		\end{align}
	\end{enumerate}
\end{sol}

\begin{example}

	\begin{figure}[H]
		\centering
		\begin{tikzpicture}[scale=1,>=Stealth,
			% Define style for the arrows along the path
			arrow on path/.style={
				postaction={
					decorate,
					decoration={
						markings,
						mark=at position #1 with {\arrow{>}}
					}
				}
			}]
		
			% Set unit length
			\def\d{1cm} % Change 1cm to your unit length if different
		
			% Draw axes
			\draw[->] (-0.5,0) -- (5*\d,0) node[anchor=north] {$x$};
			\draw[->] (0,-0.5) -- (0,5*\d) node[anchor=east] {$y$};
		
			% Define points
			\coordinate (A) at (0,0);
			\coordinate (B) at (4*\d,4*\d);
			\coordinate (C) at (4*\d,0);
		
			% Draw path segments
			% Segment C_1
			\draw[arrow on path=0.5,thick] (A) -- node[above left] {$C_1$} (B);
		
			% Segment C_2
			\draw[arrow on path=0.5,thick] (B) -- (C);
			\draw[arrow on path=0.5,thick] (C) -- node[below] {$C_2$} (A);
		
			% Label points
			\foreach \point/\position in {A/below left, B/above, C/below right}
				\draw (\point) node[\position] {$\point$};
		\end{tikzpicture}
	\end{figure}

	A particle, mass $m$, is pulled across a horizontal force with coefficient of kinetic friction $\mu_k$. First it is posted along $C_1$, then pushed along $C_2$, bringing it back to where it started. Compute $W$ along total path by friction.

	where $B = (4d, 4d)$.
\end{example}

\begin{sol}
	Find $W_{C_1}$:

	We know that

	\begin{equation}
		\vec F_{fk} = \mu_k mg \left[-\alpha (\hat x + \hat y)\right]
	\end{equation}

	then

	\begin{equation}
		\vec F_{fk} \cdot \D \vec l = \left[-\frac{\mu_k mg}{\sqrt{2}} (\hat x + \hat y)\right] \cdot \left[\frac{1}{\sqrt{2}} \D s \hat x + \frac{1}{\sqrt{2}} \D s \hat y\right] = -\mu_k mg \D s
	\end{equation}

	Then we calculate

	\begin{align}
		W_{C_1} &= \int_{C_1} \vec F_{fk} \cdot \D \vec l\\
		&= -4\sqrt{2}\mu_k mgd
	\end{align}

	$W_{C_2}$ is trivial.

	\begin{align}
		W_{C_2} &= -8 \mu_k mgd
	\end{align}

	If we sum it up, we realize that it is not equal to 0.

	\textbf{Conclusion:}

	Friction is not a nonconservative force.
\end{sol}

\section{Conservative Force Fields}

\begin{theorem}
	The following statements are equivalent:
	
	\begin{enumerate}
		\item The work done by $\vec F$ is path-independent
		\item The work done by $\vec F$ along any closed path is 0.
		\begin{equation}
			\oint_C \vec F \cdot \differential \vec l = 0
		\end{equation}

		\item $\nabla \cross \vec F = 0$
		\item There exists a scalar function $u(\vec r)$ s.t. $\vec F = -\nabla u$
		
		\begin{equation}
			u(\vec r_a) - u(\vec r_b) = - \int_{\vec r_a}^{\vec r_b} \vec F \cdot \differential \vec l
		\end{equation}
	\end{enumerate}

	We call $u(\vec r)$ the \textbf{potential energy} associated with force $\vec F$. Moreover:

	\begin{equation}
		W = - \Delta u
	\end{equation}

	If $u$ exist for $\vec F$ we say $\vec F$ is a \textbf{conservative force}.
\end{theorem}

Recall $\forall \vec F, W = \Delta K$

If $\vec F$ is conservative, then $W = -\Delta u = \Delta K$

Thus,

\begin{equation}
	\Delta K + \Delta u = 0
\end{equation}

Three examples of conservative forces are:

\begin{enumerate}
	\item Constant force
	\item Spring force
	\item Central force
\end{enumerate}

\section{Different Potential Energies}

\subsection{Gravitational Potential Energy}

\begin{definition}[Gravitational Potential Energy]
	The gravitational potential energy is defined as

	\begin{equation}
		U_g(y) = mgy
	\end{equation}
\end{definition}

obtained via

\begin{align}
	U_g(y) - U_g(y_0) &= - \int_{y_0}^y \vec F_g \cdot \left(\D y \hat y\right)\\
	&= mg (y - y_0)\\
	&= mgy - mgy_0
\end{align}

Note that

\begin{itemize}
	\item It is typical to take $U_g = 0$ at $y = 0$; i.e. reference point $y_0$ is $y_0 = 0$
	\item Physically, only the difference in $u$ matters, so shifting $u$ by a consstant leaves physics unaltered.
\end{itemize}

\subsection{Spring/Elastic Potential Energy}

\begin{definition}[Spring/Elastic Potential Energy]
	The spring potential energy is defined as

	\begin{equation}
		U_s(x) = \frac{1}{2} k \Delta x^2
	\end{equation}

	but it is usually referred to as

	\begin{equation}
		U_s(x) = \frac{1}{2} k x^2
	\end{equation}

	where it is assumed $x_0 = l_0$ where $x = 0$ is rest length.
\end{definition}

We know that $\vec F_s = -k\Delta \vec r$, so we choose coordinates so that $\Delta \vec r = \Delta x \hat x = (x - l_0) \hat x$.

\begin{align}
	U_s(x) - U_s(x_0) &= - \int_{x_0}^x \vec F_s \cdot (\D x \hat x)\\
	&= +k \int_{x_0}^x (x - l_0) \D x\\
	&= \frac{1}{2} k (x - l_0)^2 - \frac{1}{2} k (x_0 - l_0)^2
\end{align}

\subsection{Central Force}

\begin{definition}[Central Force Potential Energy]
	The general form for potential energy related to central forces is

	\begin{equation} U_c(r) - U_c(r_0) = -\frac{A}{r} + \frac{A}{r_0}
	\end{equation}

	which means

	\begin{equation}
		U_c(r) = - \frac{A}{r}
	\end{equation}
\end{definition}

We have $\vec F = f(r) \hat r = - \nicefrac{A}{r^2} \hat r$

\begin{align}
	U_c(r) - U_c(r_0) &= - \int_{r_0}^r \vec F \cdot \left(\D r \hat r\right)\\
	&= - \int_{r_0}^r f(r) \D r\\
	&= A \int_{r_0}^r \frac{1}{r^2} \D r
\end{align}

Typically speaking, we take $r_0 = \infty$ s.t. $U_c(r_0 = \infty) = 0$.

\begin{remark}
	Both Newton's law of universal gravitation and Coulomb's law for electrostatics are of the form $\vec F \propto \frac{1}{r^2} \hat r \implies u \propto - \frac{1}{r}$.
\end{remark}

\section{Definition of Energy}

\begin{definition}[Mechanical Energy]
	We define

	\begin{equation}
		E = K + U \qquad \Delta E = 0
	\end{equation}

	Then, $\Delta E = 0$ a dynamical quantity that does not change in time is called ``conserved''.

	If $\vec F$ is conservative, then, energy is conserved.
\end{definition}

Now, consider a system of particles interacting only via conservative forces.

A \textbf{system} is an arbitrary division of a collection of particles declared to be either in the system or not and hence part of the environment.

Supposed there are no external forces on the system, then:

\begin{equation}
	W_\mathrm{total} = \Delta K_\mathrm{total} = - \Delta U_\mathrm{total}
\end{equation}

where \textit{total} refers to sum over all particles \& interactions.

This means, then, 

\begin{equation}
	\Delta K_\mathrm{total} + \Delta U_\mathrm{total} = 0
\end{equation}

Potential energy is energy \textit{stored in a system} due to \textit{conservative interactions} that is reversibly transmutable to other forms (i.e. kinetic energy).

In other terms, potential energy exist with regards to fields (e.g. EM fields, gravitational fields).

\begin{definition}[Law of Conservation of Mechanical Energy]
	In a closed and isolated system, all of whose internal interactions are conservative, the total mechanical energy is constant in time, or conserved, along the motion.
\end{definition}

\begin{proof}
	Consider a closed and isolated system with only a conservative force. Then, $E = K + U = \frac{1}{2}mv^2 + U(\vec r)$.

	\begin{align}
		\frac{\D E}{\D t} &= m\vec v \cdot \frac{\D \vec v}{\D t} + \nabla u \cdot \frac{\D \vec r}{\D t}\\
		&= \vec v \cdot \left[m \frac{\D \vec v}{\D t} + \nabla u\right]\\
		&= \vec v \cdot \left[\frac{\D \vec p}{\D t} - \vec F\right]
	\end{align}

	but Newton's 2nd law says $\frac{\D \vec p}{\D t} = \vec F$ i.e. $\frac{\D E}{\D t} = 0$.

	Note if $U = U(\vec r, t)$, then $\frac{\D E}{\D t} = \frac{\partial U}{\partial t}$ and so not conserved.

	Think of my dynamical law as 

	\begin{equation}
		\frac{\D \vec p}{\D t} = -\nabla U
	\end{equation}

	then if $U$ does not depend explicitly on time, $E$ is conserved. We say that law that time-translation symmetry.
\end{proof}

\begin{definition}[A Definition of Energy]
	Energy is the quantity that is constant in time because the laws of physics has time-translation symmetry.

	An extention from \href{https://en.wikipedia.org/wiki/Noether%27s_theorem}{\textbf{Noether's Theorem}}.
\end{definition}

A slight extension:

\begin{itemize}
	\item Symmetry in space/location -- conservation of momentum
	\item Symmetry in angles/rotation -- conservation of angular momentum
	\item Gauge symmetry -- electric charge
\end{itemize}

\section{Examples}

\begin{example}
	Segway to the average of a periodic function over time (i.e. $\sin(t), \cos(t)$) for integral number of periods.
\end{example}

\begin{sol}
	\begin{align}
		\overline{K} &= \frac{1}{nT} \int_0^{nT} K \D t\\
		&= \frac{1}{nT} \cdot \frac{1}{2} m \int_0^{nT} \dot x^2 \D t\\
		&= \frac{m\omega^2}{2nT}A^2 \int_0^{nT} \sin^2\left(\omega t + \phi\right)\D t\\
		&= \frac{m\omega^2}{2nT}A^2 \int_0^{nT} \frac{1}{2} \left(1 - \cos\left[2(\omega t + \phi)\right]\right)\D t\\
		&= \frac{m\omega^2}{2nT}A^2 \left[\frac{1}{2}nT + \left.\frac{\sin\left[2(\omega t + \phi)\right]}{2\omega}\right|_0^{nT}\right]\\
		&= \left(\frac{1}{2}m\omega^2 A^2\right) \cdot \frac{1}{2} + \frac{m\omega^2A^2}{4nT\omega} \left[\sin(2n\omega T + 2\phi) - \sin(2\phi)\right] \footnotemark\\
		&= \frac{1}{2} \left(\frac{1}{2}kA^2\right) = \frac{1}{2}E
	\end{align}
	\footnotetext{Note that this, if we expand with sum of angles, evaluates to 0 (the expression in the square brackets)}

	Similarly, $\overline{U_s} = \frac{1}{2} E = \overline{K}$.

	\textbf{Conclusion:}

	$K$ and $U$ are $\frac{\pi}{2}$ out of phase, and follows the above relation.
\end{sol}

\begin{example}
	Example of central force
\end{example}

\begin{sol}
	omitted
\end{sol}

\begin{example}
	Use energy methods to show the motion of a simple pendulum (mass $m$, length $l$) is simple harmonic for small angles. What is the first correction to period if $\theta$ is not small?
\end{example}

\begin{sol}
	\begin{enumerate}[a)]
		\item From the figure,\footnote{TBA}

		\begin{equation}
			U_g = mg(l - l\cos\theta) = mgl(1-\cos\theta)
		\end{equation}

		Let the initial angle be $\theta_0$, then $\theta = \theta_0 \implies K = 0$ and $U_g = mgl(1-\cos\theta)$.

		TBF

		\item Given the original equation:
		
		\begin{equation}
			\dot \theta^2 = -2\left(\frac{g}{l}\right) \left[\cos\theta_0 - \cos\theta\right]
		\end{equation}

		which becomes

		\begin{equation}
			\int_{\theta_0}^\theta \frac{\D \theta}{\sqrt{\cos\theta - \cos\theta_0}} = \sqrt{2} \sqrt{\frac{g}{l}} t
		\end{equation}

		For a period, $t = T$ change variables in integral: $\sin u \equiv \sin(\theta / 2) / \sin(\theta_0 / 2)$ and $K \equiv \sin(\theta_0 / 2)$. Then show

		\begin{equation}
			\sqrt{2} \int_0^{2\pi} \frac{\D u}{\sqrt{1 - k^2\sin^2 u}} = \sqrt{2} \sqrt{\frac{g}{l}} T = \sqrt{2} \left(\frac{2\pi}{T_0}\right) T
		\end{equation}
	\end{enumerate}
\end{sol}

	\appendix

\end{document}