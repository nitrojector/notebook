\chapter{Energy}

\section{Derivation from Newton's Laws}

Work and energy can actually be thought as a consequence of Newton's laws and is another way of looking at the motion of objects. For more on the derivation, check provided lecture notes.

\begin{theorem}[Work-Energy Theorem]
	In simple form:
	\begin{equation}
		W = \Delta K
	\end{equation}

	In general form:
	\begin{equation}
		\int_C \vec F \cdot \D \vec l = \frac{1}{2}mv^2 - \frac{1}{2}m{v_0}^2
	\end{equation}
\end{theorem}

\section{Work \& Energy}

We know $\vec F \propto \vec a \implies \vec F \parallel \D \vec r$ changes speed; $\vec F \perp \D \vec r$ changes direction.

In our process of deriving $W = \Delta K$, we turned a vector equation into a scalar equation. Hence, here we are concerned only with $\vec F_\parallel$ (moreover, if I draw trajectory I know what $\vec F_\perp$ does but not $\vec F_\parallel$)

Evidently, $\vec F_\parallel \implies \vec F \cdot \D \vec r \propto$ change in speed.

\begin{equation}
	\D \vec v \cdot \vec v \sim \frac{1}{2}\D(v^2)
\end{equation}

More precisely, the 2nd law of motion:

\begin{equation}
	\vec F \cdot \D \vec r = \frac{1}{2} m \D(v^2)
\end{equation}

Infinitesimally, this would be:

\begin{equation}
	\vec F(\vec r_j) \cdot \D \vec r_j = \frac{1}{2} m \left[{v_{j+1}}^2 - {v_j}^2\right]
\end{equation}

We can informally sum all these parts as

\begin{equation}
	\sum_{\vec r = \vec r_0}^{\vec r} \vec F(\vec r_j) \cdot \D \vec r_j = \frac{1}{2} m \left(v^2 - v_0^2\right) = \int_{\vec r_0}^{\vec r} \vec F(\vec r) \cdot \D \vec r
\end{equation}

\textbf{Conclusion}

$W$ is adding up all the infinitesimal contributions of the force tangent to the trajectory (i.e. ones that change the speed) along the trajectory from start to finish.

$K$ measures the change in speed due to $\vec F$ as required by Newton's 2nd law.

$W$ is a sum of infinitesimal scalar quantities along a curve which is a line integral.

In cartesian:

\begin{equation}
	\int \vec F(\vec r) \cdot \D \vec r = \int \vec F(x, y, z) \cdot \left[\D x \hat x + \D y \hat y + \D z \hat z\right]
\end{equation}

\subsection{Kinetic Energy}

\begin{equation}
	K \equiv \frac{1}{2}mv^2
\end{equation}

\begin{itemize}
	\item Kinetic energy is, well, energy associated with motion.

	\item $K$ is frame-dependent.
	
	\item And the units are Joules.
\end{itemize}

\subsection{Work}

\begin{equation} \label{eq:work}
	W = \int_C \vec F \cdot \D \vec l
\end{equation}

Since $W = \Delta K$ is the change in energy of our particle/system.

\begin{itemize}
	\item If $W > 0 \implies K > K_0 \implies$ gained energy/speed
	\item If $W < 0 \implies K < K_0 \implies$ loses energy/speed
	\item If $W = 0 \implies K = K_0 \implies$ no change in energy/speed
\end{itemize}

Work is the energy transferred into/out-of a system by mechanical means (i.e. application of a fore over a displacement)

We say ``Work is done by force $\vec F$ along the path'' when writing \cref*{eq:work}.

It is useful to measure rate at which work is done -- called the \textbf{power} [units: Watts 1 W = 1 J/s]

\begin{align} 
	\frac{\D W}{\D t} &= \frac{\D}{\D t}(W)\\
	&= \frac{\D}{\D t} \left[\int_{\vec r_0}^{\vec r} \vec F(\vec r') \cdot \D \vec r'\right]\\
	&= \frac{\D}{\D t} \left[\int_{t_0}^t \vec F(\vec v) \cdot \vec v(t') \D t'\right]
\end{align}

And so we have:

\begin{equation} \label{eq:power}
	\frac{\D W}{\D t} = \vec F \cdot \vec V
\end{equation}

in this case, $\D W$ refers to an infinitesimal amount of work instead of it's change. It is technically $\mathrm{d}\hspace*{-0.08em}\bar{}\hspace*{0.1em} W$ -- inexact differential.

\begin{example}
	Consider constant force $\vec F = F_0 \hat n$, $F_0 = \mathrm{const}$, $\hat n$ constant unit vector.

	Compute work done over displacement $\Delta \vec r = \vec f - \vec f_0$
\end{example}

\begin{sol}
	\begin{equation}
		W = \vec F \cdot \Delta \vec r
	\end{equation}
\end{sol}

\begin{example}
	Consider a central force, $\vec F(\vec r) = f(r) \hat r$ work in 2-dim

	\begin{enumerate}[(a)]
		\item Show W is independent of path
		\item Let $f(r) = - A/r^2$ for $A > 0$ a constant. Find $v(r)$ if $v(r = r_0) = 0$.
	\end{enumerate}
\end{example}

\begin{sol}
	\begin{enumerate}[(a)]
		\item Work in polar coords
		
		$\D \vec l = \D r \hat r + r \D \theta \hat \theta$

		Thus, 

		\begin{align}
			W &= \int_C \vec F cdot \D \vec l\\
			&= \int_C f(r) \hat r \cdot (\D r \hat r + r \D \theta \hat \theta)\\
			&= \int_{r_0}^r f(r) \D r
		\end{align}

		Since it only requires the endpoints, it does not require the path, it is independent.

		\item $f(r) = -\frac{A}{r^2}$
		
		\begin{equation}
			W = \int_{r_0}^r -\frac{A}{r^2} \D r = A(\frac{1}{r} - \frac{1}{r_0})
		\end{equation}

		We can apply $W = \Delta K$

		\begin{align}
			A\left(\frac{1}{r} - \frac{1}{r_0}\right) &= \frac{1}{2}mv^2(r)\\
			\Aboxed{v(r) &= \pm \sqrt{\frac{2A}{m}\left(\frac{1}{r} - \frac{1}{r_0}\right)}}
		\end{align}
	\end{enumerate}
\end{sol}

\begin{example}

	\begin{figure}[H]
		\centering
		\begin{tikzpicture}[scale=1,>=Stealth,
			% Define style for the arrows along the path
			arrow on path/.style={
				postaction={
					decorate,
					decoration={
						markings,
						mark=at position #1 with {\arrow{>}}
					}
				}
			}]
		
			% Set unit length
			\def\d{1cm} % Change 1cm to your unit length if different
		
			% Draw axes
			\draw[->] (-0.5,0) -- (5*\d,0) node[anchor=north] {$x$};
			\draw[->] (0,-0.5) -- (0,5*\d) node[anchor=east] {$y$};
		
			% Define points
			\coordinate (A) at (0,0);
			\coordinate (B) at (4*\d,4*\d);
			\coordinate (C) at (4*\d,0);
		
			% Draw path segments
			% Segment C_1
			\draw[arrow on path=0.5,thick] (A) -- node[above left] {$C_1$} (B);
		
			% Segment C_2
			\draw[arrow on path=0.5,thick] (B) -- (C);
			\draw[arrow on path=0.5,thick] (C) -- node[below] {$C_2$} (A);
		
			% Label points
			\foreach \point/\position in {A/below left, B/above, C/below right}
				\draw (\point) node[\position] {$\point$};
		\end{tikzpicture}
	\end{figure}

	A particle, mass $m$, is pulled across a horizontal force with coefficient of kinetic friction $\mu_k$. First it is posted along $C_1$, then pushed along $C_2$, bringing it back to where it started. Compute $W$ along total path by friction.

	where $B = (4d, 4d)$.
\end{example}

\begin{sol}
	Find $W_{C_1}$:

	We know that

	\begin{equation}
		\vec F_{fk} = \mu_k mg \left[-\alpha (\hat x + \hat y)\right]
	\end{equation}

	then

	\begin{equation}
		\vec F_{fk} \cdot \D \vec l = \left[-\frac{\mu_k mg}{\sqrt{2}} (\hat x + \hat y)\right] \cdot \left[\frac{1}{\sqrt{2}} \D s \hat x + \frac{1}{\sqrt{2}} \D s \hat y\right] = -\mu_k mg \D s
	\end{equation}

	Then we calculate

	\begin{align}
		W_{C_1} &= \int_{C_1} \vec F_{fk} \cdot \D \vec l\\
		&= -4\sqrt{2}\mu_k mgd
	\end{align}

	$W_{C_2}$ is trivial.

	\begin{align}
		W_{C_2} &= -8 \mu_k mgd
	\end{align}

	If we sum it up, we realize that it is not equal to 0.

	\textbf{Conclusion:}

	Friction is not a nonconservative force.
\end{sol}

\section{Conservative Force Fields}

\begin{theorem}
	The following statements are equivalent:
	
	\begin{enumerate}
		\item The work done by $\vec F$ is path-independent
		\item The work done by $\vec F$ along any closed path is 0.
		\begin{equation}
			\oint_C \vec F \cdot \differential \vec l = 0
		\end{equation}

		\item $\nabla \cross \vec F = 0$
		\item There exists a scalar function $u(\vec r)$ s.t. $\vec F = -\nabla u$
		
		\begin{equation}
			u(\vec r_a) - u(\vec r_b) = - \int_{\vec r_a}^{\vec r_b} \vec F \cdot \differential \vec l
		\end{equation}
	\end{enumerate}

	We call $u(\vec r)$ the \textbf{potential energy} associated with force $\vec F$. Moreover:

	\begin{equation}
		W = - \Delta u
	\end{equation}

	If $u$ exist for $\vec F$ we say $\vec F$ is a \textbf{conservative force}.
\end{theorem}

Recall $\forall \vec F, W = \Delta K$

If $\vec F$ is conservative, then $W = -\Delta u = \Delta K$

Thus,

\begin{equation}
	\Delta K + \Delta u = 0
\end{equation}

Three examples of conservative forces are:

\begin{enumerate}
	\item Constant force
	\item Spring force
	\item Central force
\end{enumerate}

\section{Different Potential Energies}

\subsection{Gravitational Potential Energy}

\begin{definition}[Gravitational Potential Energy]
	The gravitational potential energy is defined as

	\begin{equation}
		U_g(y) = mgy
	\end{equation}
\end{definition}

obtained via

\begin{align}
	U_g(y) - U_g(y_0) &= - \int_{y_0}^y \vec F_g \cdot \left(\D y \hat y\right)\\
	&= mg (y - y_0)\\
	&= mgy - mgy_0
\end{align}

Note that

\begin{itemize}
	\item It is typical to take $U_g = 0$ at $y = 0$; i.e. reference point $y_0$ is $y_0 = 0$
	\item Physically, only the difference in $u$ matters, so shifting $u$ by a consstant leaves physics unaltered.
\end{itemize}

\subsection{Spring/Elastic Potential Energy}

\begin{definition}[Spring/Elastic Potential Energy]
	The spring potential energy is defined as

	\begin{equation}
		U_s(x) = \frac{1}{2} k \Delta x^2
	\end{equation}

	but it is usually referred to as

	\begin{equation}
		U_s(x) = \frac{1}{2} k x^2
	\end{equation}

	where it is assumed $x_0 = l_0$ where $x = 0$ is rest length.
\end{definition}

We know that $\vec F_s = -k\Delta \vec r$, so we choose coordinates so that $\Delta \vec r = \Delta x \hat x = (x - l_0) \hat x$.

\begin{align}
	U_s(x) - U_s(x_0) &= - \int_{x_0}^x \vec F_s \cdot (\D x \hat x)\\
	&= +k \int_{x_0}^x (x - l_0) \D x\\
	&= \frac{1}{2} k (x - l_0)^2 - \frac{1}{2} k (x_0 - l_0)^2
\end{align}

\subsection{Central Force}

\begin{definition}[Central Force Potential Energy]
	The general form for potential energy related to central forces is

	\begin{equation} U_c(r) - U_c(r_0) = -\frac{A}{r} + \frac{A}{r_0}
	\end{equation}

	which means

	\begin{equation}
		U_c(r) = - \frac{A}{r}
	\end{equation}
\end{definition}

We have $\vec F = f(r) \hat r = - \nicefrac{A}{r^2} \hat r$

\begin{align}
	U_c(r) - U_c(r_0) &= - \int_{r_0}^r \vec F \cdot \left(\D r \hat r\right)\\
	&= - \int_{r_0}^r f(r) \D r\\
	&= A \int_{r_0}^r \frac{1}{r^2} \D r
\end{align}

Typically speaking, we take $r_0 = \infty$ s.t. $U_c(r_0 = \infty) = 0$.

\begin{remark}
	Both Newton's law of universal gravitation and Coulomb's law for electrostatics are of the form $\vec F \propto \frac{1}{r^2} \hat r \implies u \propto - \frac{1}{r}$.
\end{remark}

\section{Definition of Energy}

\begin{definition}[Mechanical Energy]
	We define

	\begin{equation}
		E = K + U \qquad \Delta E = 0
	\end{equation}

	Then, $\Delta E = 0$ a dynamical quantity that does not change in time is called ``conserved''.

	If $\vec F$ is conservative, then, energy is conserved.
\end{definition}

Now, consider a system of particles interacting only via conservative forces.

A \textbf{system} is an arbitrary division of a collection of particles declared to be either in the system or not and hence part of the environment.

Supposed there are no external forces on the system, then:

\begin{equation}
	W_\mathrm{total} = \Delta K_\mathrm{total} = - \Delta U_\mathrm{total}
\end{equation}

where \textit{total} refers to sum over all particles \& interactions.

This means, then, 

\begin{equation}
	\Delta K_\mathrm{total} + \Delta U_\mathrm{total} = 0
\end{equation}

Potential energy is energy \textit{stored in a system} due to \textit{conservative interactions} that is reversibly transmutable to other forms (i.e. kinetic energy).

In other terms, potential energy exist with regards to fields (e.g. EM fields, gravitational fields).

\begin{definition}[Law of Conservation of Mechanical Energy]
	In a closed and isolated system, all of whose internal interactions are conservative, the total mechanical energy is constant in time, or conserved, along the motion.
\end{definition}

\begin{proof}
	Consider a closed and isolated system with only a conservative force. Then, $E = K + U = \frac{1}{2}mv^2 + U(\vec r)$.

	\begin{align}
		\frac{\D E}{\D t} &= m\vec v \cdot \frac{\D \vec v}{\D t} + \nabla u \cdot \frac{\D \vec r}{\D t}\\
		&= \vec v \cdot \left[m \frac{\D \vec v}{\D t} + \nabla u\right]\\
		&= \vec v \cdot \left[\frac{\D \vec p}{\D t} - \vec F\right]
	\end{align}

	but Newton's 2nd law says $\frac{\D \vec p}{\D t} = \vec F$ i.e. $\frac{\D E}{\D t} = 0$.

	Note if $U = U(\vec r, t)$, then $\frac{\D E}{\D t} = \frac{\partial U}{\partial t}$ and so not conserved.

	Think of my dynamical law as 

	\begin{equation}
		\frac{\D \vec p}{\D t} = -\nabla U
	\end{equation}

	then if $U$ does not depend explicitly on time, $E$ is conserved. We say that law that time-translation symmetry.
\end{proof}

\begin{definition}[A Definition of Energy]
	Energy is the quantity that is constant in time because the laws of physics has time-translation symmetry.

	An extention from \href{https://en.wikipedia.org/wiki/Noether%27s_theorem}{\textbf{Noether's Theorem}}.
\end{definition}

A slight extension:

\begin{itemize}
	\item Symmetry in space/location -- conservation of momentum
	\item Symmetry in angles/rotation -- conservation of angular momentum
	\item Gauge symmetry -- electric charge
\end{itemize}

\section{Examples}

\begin{example}
	Segway to the average of a periodic function over time (i.e. $\sin(t), \cos(t)$) for integral number of periods.
\end{example}

\begin{sol}
	\begin{align}
		\overline{K} &= \frac{1}{nT} \int_0^{nT} K \D t\\
		&= \frac{1}{nT} \cdot \frac{1}{2} m \int_0^{nT} \dot x^2 \D t\\
		&= \frac{m\omega^2}{2nT}A^2 \int_0^{nT} \sin^2\left(\omega t + \phi\right)\D t\\
		&= \frac{m\omega^2}{2nT}A^2 \int_0^{nT} \frac{1}{2} \left(1 - \cos\left[2(\omega t + \phi)\right]\right)\D t\\
		&= \frac{m\omega^2}{2nT}A^2 \left[\frac{1}{2}nT + \left.\frac{\sin\left[2(\omega t + \phi)\right]}{2\omega}\right|_0^{nT}\right]\\
		&= \left(\frac{1}{2}m\omega^2 A^2\right) \cdot \frac{1}{2} + \frac{m\omega^2A^2}{4nT\omega} \left[\sin(2n\omega T + 2\phi) - \sin(2\phi)\right] \footnotemark\\
		&= \frac{1}{2} \left(\frac{1}{2}kA^2\right) = \frac{1}{2}E
	\end{align}
	\footnotetext{Note that this, if we expand with sum of angles, evaluates to 0 (the expression in the square brackets)}

	Similarly, $\overline{U_s} = \frac{1}{2} E = \overline{K}$.

	\textbf{Conclusion:}

	$K$ and $U$ are $\frac{\pi}{2}$ out of phase, and follows the above relation.
\end{sol}

\begin{example}
	Example of central force
\end{example}

\begin{sol}
	omitted
\end{sol}

\begin{example}
	Use energy methods to show the motion of a simple pendulum (mass $m$, length $l$) is simple harmonic for small angles. What is the first correction to period if $\theta$ is not small?
\end{example}

\begin{sol}
	\begin{enumerate}[a)]
		\item From the figure,\footnote{TBA}

		\begin{equation}
			U_g = mg(l - l\cos\theta) = mgl(1-\cos\theta)
		\end{equation}

		Let the initial angle be $\theta_0$, then $\theta = \theta_0 \implies K = 0$ and $U_g = mgl(1-\cos\theta)$.

		TBF

		\item Given the original equation:
		
		\begin{equation}
			\dot \theta^2 = -2\left(\frac{g}{l}\right) \left[\cos\theta_0 - \cos\theta\right]
		\end{equation}

		which becomes

		\begin{equation}
			\int_{\theta_0}^\theta \frac{\D \theta}{\sqrt{\cos\theta - \cos\theta_0}} = \sqrt{2} \sqrt{\frac{g}{l}} t
		\end{equation}

		For a period, $t = T$ change variables in integral: $\sin u \equiv \sin(\theta / 2) / \sin(\theta_0 / 2)$ and $K \equiv \sin(\theta_0 / 2)$. Then show

		\begin{equation}
			\sqrt{2} \int_0^{2\pi} \frac{\D u}{\sqrt{1 - k^2\sin^2 u}} = \sqrt{2} \sqrt{\frac{g}{l}} T = \sqrt{2} \left(\frac{2\pi}{T_0}\right) T
		\end{equation}
	\end{enumerate}
\end{sol}