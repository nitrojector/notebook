\chapter{Rigid Body Motion}

\section{Introduction}

When we talk about rigid body motion, at least at our current level, will be subject to several constraints:

\begin{itemize}
	\item The axis of rotation will be fixed in direction, though it may translate.
	\item The body is rigid and thus do not deform.
\end{itemize}

\begin{theorem}[Chasles' Theorem]
	Most general rigid body displacement can be produced by a translation along a line (called its screw axis or Mozzi axis) followed (or preceded) by a rotation about an axis parallel to that line.
\end{theorem}

\section{Rotational Kinematics}

% !!TBA

We can categorize rotation into two categories:

\begin{description}
	\item[Orbiting] Rotation about an axis external to the body
	\item[Spinning] Rotation about axis internal to the body
\end{description}

Consider coordinates s.t. the axis of rotation of rigid body is the $z$-axis. Then, the rotation view down the $z$-axis is either clockwise (CW) or counterclockwise (CCW).

Angular speed $\omega$ measures rate of rotation. To include direction of rotation, we promote $\omega$ to a vector: $\vec\omega$.

The direction of $\vec\omega$ is along axis of rotation (i.e. $\pm z$-axis) s.t. rotation follows the RHR.

We do a similar thing with $\alpha$ to $\vec\alpha$.

\begin{remark}
	For fixed axis rotation i.e. 1-dim motion

	\begin{itemize}
		\item If $\alpha$ and $\omega$ have the same sign, the body is rotating faster
		\item If they have opposite signs, then the body is rotating slower.
	\end{itemize}

	Also, technically $\vec\omega$ and $\vec\alpha$ are ``pseudovectors'' and if we change around the axes, then there would be a sign change on the vector.
\end{remark}

\section{Rotational Dynamics}

The rotational analogue of force is \textbf{torque}, $\vec\tau$, s.t.

\begin{equation}
	\vec \tau = \vec r \cross \vec F
\end{equation}

where $\vec r$ is the position vector

\begin{description}
	\item[orbiting] $\vec r$ is from origin to particle (which is the point of application of the force). If we change the origin, then we are changing the torque.
	\item[spinning] $\vec r$ is from origin to point of application of the force. 
\end{description}

and $\vec F$ is the force.

Thus, in conclusion, $\vec tau$ depends on the point of origin, i.e. the point about which it is computed (via $\vec r$).

Typically we choose the origin to lie on axis of rotation.

A body in \textbf{static equilibrium} requires that $\sum \vec F = 0$ and $\sum \vec\tau = 0$.

\begin{example}
	Show that the torque due to gravity about any point always acts at the center of mass: $\vec \tau = \vec R \cross \vec F_g$, where $\vec R$ is the center of mass (measured from the point) and $\vec F_g$ is the weight.
\end{example}

\begin{sol}
	Torque due to gravity on $m_j$: $\vec\tau_j = \vec r_j \cross \vec F_{gj} = \vec r_j \cross m_j \vec g$.

	Net torque on body is then

	\begin{equation}
		\vec \tau_g = \sum_j \vec r_j \cross m_j \vec g
	\end{equation}

	Given that

	\begin{equation}
		\vec R = \frac{1}{M} \sum m_j \cdot \vec r_j
	\end{equation}

	we can rearrange

	\begin{align}
		\vec \tau_g &= \sum \vec r_j \cross m_j \vec g\\
		&= \sum_j (m_j \vec r_j) \cross \vec g\\
		&= M\vec R \cross \vec g\\
		&= \vec R \cross \vec F_g
	\end{align}

	Thus, in any coordinates, the torque due to gravity acts at the center of mass of the object, i.e. we can take the point of application of $\vec F_g$ to be at C.O.M. (sometimes center of mass is called center of gravity)
\end{sol}

\begin{example}
	A uniform of length $\frac{\pi R}{2}$ is bent into a quadrant of a circle of radius $R$.

	It touches ground and wall -- there is no friction with the wall but there is friction with the ground. 

	The rod is in static equilibrium. Find the force the wall exerts on the rod.
\end{example}

\begin{sol}
	Static equilibrium implies that $\sum \vec \tau = 0$ and $\sum \vec F = 0$. We choose to compute $\vec \tau$ about center point with the floor.

	\begin{equation}
		\vec\tau = \vec r_w \cross \vec F_{NW} + \vec R \cross \vec F_g
	\end{equation}

	We can take care of direction of rotation automatically.

	We weite $\abs{\vec r \cross \vec F} = r_\perp F = r F_\perp = rF\sin\phi$

	Taking the perpendicular component of $r$ is easiest for this problem. Thus, we need the $y$-component of $\vec r_w$ and need the $x$-component of $\vec R$.

	The $y$-component of $\vec F_W = R$

	The $x$-component of $\vec R = R - R\cos\frac{\pi}{4}$

	Taking everything together:

	\begin{align}
		0 &= -RF_{NW} + R(1 - \frac{\sqrt{2}}{2})Mg\\
		F_{NW} &= \boxed{(1 - \frac{\sqrt{2}}{2})Mg}
	\end{align}
\end{sol}

\section{Angular Momentum \& Rotational 2nd Law}

We have established rotational analogue of force -- torque.

We know the linear 2nd law and given $\vec \tau = \vec r \cross \vec F$, we can try to $\vec r \cross \text{2nd law}$.

\begin{align}
	\vec \tau &= \vec r \cross \vec F\\
	&= \vec r \cross \frac{\D \vec p}{\D t}\\
	&= \frac{\D}{\D t}\left(\vec r \cross \vec p\right) - \frac{\D \vec r}{\D t} \cross \vec p\\
	&= \frac{\D}{\D t}\left(\vec r \cross \vec p\right)
\end{align}

Thus, we define the \textbf{angular momentum}

\begin{definition}[Angular Momentum \& Rotational 2nd Law]
	We defined angular momentum to be

	\begin{equation}
		\vec L = \vec r \cross \vec p
	\end{equation}

	and the 2nd law to be

	\begin{equation}
		\vec \tau = \frac{\D \vec L}{\D t}
	\end{equation}
\end{definition}

\begin{remark}
	With regards to angular momentum
	\begin{itemize}
		\item Depends on the point about which it is computed/origin of coordinates
		\item A particle moving in a straight line may have angular momentum
		\item Direction of $\vec L$ is $\perp$ to $\vec r$ and $\vec p$, thus perpendicular plane of motion. The positive and negative sense of $\vec L$ is similar to what we mentioned with $\vec \omega$ and $\vec \alpha$.
		\item The magnitude is $L = rp\sin\phi = r_\perp p = rp_\perp$
		\item A userful expression to recognize:
		
		\begin{align}
			\vec L = \vec r \cross \vec p &= \langle x, y \rangle \cross \langle p_x, p_y \rangle\\
			&= (xp_y - yp_x) \hat z\\
			&= m (x\dot y - y \dot x) \hat z
		\end{align}
	\end{itemize}

	Thus,

	\textit{angular momentum gives a measure of the sense of rotation of the motion of a particle in given coordinates.}
\end{remark}

\begin{example}
	Consider a block of mass $m$ sliding along a straight line, say $+x$-axis, with velocity $\vec v = v \hat x$. Suppose it is subject to a friction force $\vec F_f = - f \hat x$.

	Find:

	\begin{enumerate}
		\item $\vec L_A$ and $\vec \tau_A$
		\item $\vec L_B$ and $\vec \tau_B$
		\item Show $\vec \tau = \frac{\D \vec L}{\D t}$ for both A and B.
	\end{enumerate}
\end{example}

\begin{sol}
	\begin{enumerate}
		\item $\vec L_A = (x \hat x) \cross (mv \hat x) = 0$
		
		and $\vec \tau_A = (x \hat x) \cross (- f \hat x) = 0$

		Thus it is obvious that part 3 is true for this case.

		\item In this case $\vec r_B = x \hat x - l \hat y$.

		Hence, 
		\begin{align}
			\vec L_B = \vec r_B \cross \vec p &= (x \hat x - l \hat y) \cross (mv \hat x)\\
			&= mvl\hat z\\
			\vec \tau_B = \vec r_B \cross \vec F_f &= - fl \hat z
		\end{align}

		\item We tested for part 1 (A), for B:
		
		\begin{align}
			\frac{\D \vec L_B}{\D t} &= ml \frac{\D v}{\D t} \hat z = l \frac{\D (mv)}{\D t} \hat z = l \frac{\D p}{\D t} \hat z\\
			\vec F &= \frac{\D \vec p}{\D t} \implies - f = \frac{\D p}{\D t}
		\end{align}

		\begin{equation}
			\frac{\D \vec L_B}{\D t} = l (-f) \hat z
		\end{equation}
	\end{enumerate}
\end{sol}

\section{Angular Momentum}

Recall the rotational 2nd law, where

\begin{equation}
	\sum \vec \tau = \frac{\D \vec L}{\D t}
\end{equation}

where

\begin{definition}[Angular Momentum]
	\begin{equation}
		\vec L = \vec r \cross \vec p
	\end{equation}

	is defined to be the angular momentum.
\end{definition}

For a fixed axis rigid body rotation (i.e. spinning), $\vec L$ takes a simplified form. All points within a rigid body have some $\omega$ since it is a rigid body. The angular momentum of a particle $i$ on the rigid body is

\begin{equation}
	\vec L_i = \vec r_i \cross \vec p_i \implies \left(\vec L_i\right)_z = \left(\vec r_i \cross \vec p_i \right)_z
\end{equation}

This means that (if we consider the linear momentum to tangential with radial $r_i$, i.e. $\vec r_i \perp \vec p_i$)

\begin{equation}
	L_i = s_i m_i v_i = s_i m_i (\omega s_i) = m_i {s_i}^2 \omega
\end{equation}

This is then defined to be the \textbf{moment of inertia} $I$ for a point particle.

\begin{definition}[Moment of Inertia]
	\begin{equation}
		I_\mathrm{body} = \sum_i m_i {s_i}^2
	\end{equation}

	and to get a continuum description of body, we get

	\begin{equation}
		I = \int s^2 \D m
	\end{equation}
\end{definition}

Now, coming back to angular momentum, we then have

\begin{definition}
	\begin{equation}
		L = I\omega
	\end{equation}

	\begin{enumerate}
		\item Which is equivalent expression for $L$ for fixed axis rigid body rotation.
		\item Analogous to $p = mv$ with $I \leftrightarrow m, \omega \leftrightarrow v$.
	\end{enumerate}
\end{definition}

\section{Moment of Inertia}

\begin{enumerate}
	\item $I$ is a measure of rotational inertia... rotational analogue of mass. Greater $I$ the greater ``resistance'' of body to changes in its rotational motion.
	\item $I$ depends on axis of rotation (via $s^2$ factor), just like how $\tau$ and $L$ do as well. However, the relationships $L = I\omega$ and $\tau = \frac{\D l}{\D t}$ are \textbf{frame independent}.
	\item $I$ depends not only on the mass, but how it is distributed within the body. If most of the mass is concentrated near axis of rotation, then $I$ is smaller compared to more mass concentrated further away.
	\item Quantity of the form $\int s^n \D m$ is a ``moment'', so inertia is a ``second moment'' because $s^2$.
	\item For non-fixed axis of rotation (rigid body), turns out that $v = \omega r$ no longer applies.

	Instead, consider $\vec r_i$ position of mass $m_i$ relative to center of mass of the body.

	Note that:

	\begin{equation}
		\dot{\vec r_i} = \vec \omega \cross \vec r_i \implies \vec v = \vec \omega \cross \vec r
	\end{equation}

	To compute $\vec L$ with respect to C.O.M.

	\begin{align}
		\vec L_\mathrm{CM} &= \sum_i \vec r_i \cross \vec p_i = \sum_i \vec r_i \cross \left(m_i \dot{\vec r_i}\right) = \sum_i \vec r_i \cross m_i \left(\vec \omega \cross \vec r_i\right)\\
		\vec L_\mathrm{CM} &= \sum_i m_i \left(\vec r_i \cross \vec \omega \cross \vec r_i\right) = \sum_i m_i \left[{r_i}^2 \vec \omega - (\vec \omega \cdot \vec r_i) \vec r_i\right]\\
		&= \sum_i \left[m_i {r_i}^2 \vec \omega - (\vec \omega \cdot \vec r_i) m_i \vec r_i\right]\footnotemark\\
		&= I\omega
	\end{align}
	\footnotetext{Because the second term is zero for fixed axis rotation with $r_i = s_i$}
\end{enumerate}

A side note, moment of inertia $I$ can be written as a matrix multiplication with some $\vec \omega$.

\begin{equation}
	I = \begin{pmatrix}
		I_{xx} & I_{xy} & I_{xz}\\
		I_{yx} & I_{yy} & I_{yz}\\
		I_{zx} & I_{zy} & I_{zz}
	\end{pmatrix}
\end{equation}

But, $I$ is actually also not a matrix but rather a rank 2 tensor -- the \textbf{moment of inertia tensor}:

\begin{equation}
	I = \sum \left[m_i {r_i}^2 E_3 - m_i \vec r_i \otimes \vec r_i\right]
\end{equation}

where $E_3$ is the 3-by-3 identity matrix.

\subsection{Parallel Axis Theorem}

\begin{theorem}
	The moment of inertia $I$ about an axis parallel to an axis through the center of mass, $I_0$, a distance $d$ apart, is given by

	\begin{equation}
		I = I_0 + Md^2
	\end{equation}
\end{theorem}

\section{Dynamics of Fixed Axis Rigid Body Motion}

In this case, $L = I\omega$ with $I = \text{const}$. Therefore, rotational 2nd law may be written as

\begin{equation}
	\sum \tau = \frac{\D L}{\D t} = I \frac{\D \omega}{\D t} \implies \sum \tau = I \alpha
\end{equation}

%!! TBA Apr 2 Lecture

\begin{remark}
	Note that Chasles' Theorem still applies when the center of mass is acceleration. That is to say the motion in the angular aspect is still valid given the net torque exerted on the object.
\end{remark}

\section{Examples of Rotational Motion}

\subsection{Physical Pendulum}

%! TBA

\subsection{Dynamics of Translation \& Rotation}

%! TBA

\subsection{Examples}

\begin{example}
	Ramp mass pully example.
\end{example}

\begin{example}
	A ball of mass $M$ and radius $b$ rolls without slipping.
\end{example}

\begin{sol}
	\begin{align}
		L_z &= \pm I_\mathrm{cm} \omega + (\vec R \cross M\vec V)_z\\
		&=  - I_\mathrm{cm} \omega + \left[(x \hat x + b \hat y) \cross M (v \hat x)\right]_z\\
		&= - \left(\frac{2}{5} Mb^2\right) \omega + \left[-Mvb \hat z\right]_z\\
		&= - \frac{2}{5} Mb^2 \omega - M(\omega b) b\\
		&= -\left(\frac{2}{5} + 1\right) Mb^2\omega\\
		\Aboxed{L_z &= - \frac{7}{5} Mb^2 \omega}
	\end{align}
\end{sol}

\begin{example}
	Disk of mass $M$ and radius $b$ pulled by constant force $F$ (at the bottom) slides on ice without friction. What is its motion?
\end{example}

\begin{sol}
	Compute $\tau_z = \tau_0 + (\vec R \cross \vec F)_z$

	Here $\tau_\mathrm{cm} = + I_\mathrm{cm} \alpha  = \frac{1}{2} Mb^2 \alpha = bF \implies \alpha = \frac{2F}{Mb}$

	\begin{equation}
		(\vec R \cross \vec F)_z = \left[(x \hat x + b \hat y) \cross (F \hat x)\right] = -bF
	\end{equation}

	Thus, $\tau_z = bF - bF = 0$. So the torque with respect to the origin is 0.

	For angular momentum

	\begin{equation}
		L_z = I_\mathrm{cm} \omega + (\vec R \cross M \vec V)_z = \frac{1}{2} Mb^2 \omega - Mvb
	\end{equation}

	and

	\begin{align}
		\frac{\D L_z}{\D t} &= \frac{1}{2} Mb^2 \alpha - Mba = \tau_z = 0\\
		\frac{1}{2} Mb^2 (\frac{2F}{Mb}) - Mba &= bF - Mba \implies F = Ma
	\end{align}
\end{sol}

\section{Collection of Particles}

Our rotational 2nd law states that

\begin{equation}
	\sum \vec \tau = \frac{\D \vec L}{\D t}
\end{equation}

Consider a system of $N$ particles, with positions $\vec r_i$ subject to forces $\vec F_i = \vec F_i^\mathrm{ext} + \sum_{j\neq i} \vec F_{ji}$ Their momenta are $\vec P_i$.

Compute $\vec\tau_i$:

\begin{equation}
	\vec\tau_i = \vec r_i \cross \vec F_i = \vec r_i \cross \vec F_i^\mathrm{ext} + \vec r_i \cross \sum_{j\neq i} \vec F_{ji}
\end{equation}

The total torque is

\begin{equation}
	\vec\tau = \sum\vec\tau_i = \sum \vec r_i \cross \vec F_i^\mathrm{ext} + \sum_i \vec r_i \cross \left(\sum_{j\neq i} \vec F_{ji}\right)
\end{equation}

For our second term, consider torque on $i$-th particle due to particke $k$

\begin{equation}
	\vec\tau_{k_i} = \vec r_i \cross \vec F_{k_i}
\end{equation}

now consider torque on $k$ due to $i$:

\begin{equation}
	\vec\tau_{ik} = \vec r_{k} \cross \vec F_{ik}
\end{equation}

Now, with Newton's 3rd law $\vec F_{ik} = \vec F_{ki}$, so:

\begin{align}
	\vec\tau_{ki} + \vec\tau_{ik} &= \vec r_i \cross \vec F_{ki} + \vec r_k \cross \vec F_{ik}\\
	&= (\vec r_i - \vec r_k) \vec F_{ki}
\end{align}


This expression is not immediately 0, however, if the force $\vec F_{ik}$ is in the direction connecting the two objects, then the torque sums up to 0.

Consequently we have is that

\begin{equation}
	\vec \tau = \vec\tau_\mathrm{ext} + \frac{1}{2} \sum_i \sum_j (\vec r_i - \vec r_j) \cross \vec F_{ji} 
\end{equation}

If all internal forces are central, the second term is zero and (this seems reasonable for a rigid body consisting of $n$ particles, but is not so reasonable for free particles.)

\begin{equation}
	\vec\tau = \vec\tau_\mathrm{ext}
\end{equation}

If we consider the right hand side of the 2nd law

\begin{align}
	L_i &= \vec r_i \cross \vec P_i\\
	&= \sum_i \vec r_i \cross \vec P_i\\
	&= \sum \vec L_i\\
	&= \vec L_\mathrm{tot}
\end{align}

So, if $\vec\tau_\mathrm{ext} = 0$, we have that

\begin{equation}
	\frac{\D \vec L_\mathrm{tot}}{\D t} = 0
\end{equation}

Thus, $\vec L_\mathrm{tot} = \text{const}$

\begin{theorem}[Law of Conservation of Angular Momentum]
	If the total external torque on a system is zero, then the total angular momentum of the system is conserved:

	\begin{equation}
		\Delta \vec L_\mathrm{tot} = 0
	\end{equation}

	\begin{remark}
		However, this is assuming all pairwise internal forces are central
	\end{remark}
\end{theorem}

Recall $\vec F = -\nabla u$. Consider spherical coordinates: $u(r,\theta,\phi)$.

%! Is rotational energy conserved for a wheel that is slipping
%! in 7.13 are we not consider the rotational momentum wrt cm?

\section{Rotational Energy}

%! Derivations are omitted

\begin{definition}
	We have our new form of kinetic energy as

	\begin{equation}
		K = \frac{1}{2} I_\mathrm{cm} \omega^2 + \frac{1}{2} M {V_\mathrm{cm}}^2
	\end{equation}

	and rotational KE

	\begin{equation}
		K_\mathrm{rot} = \frac{1}{2} I \omega^2
	\end{equation}
\end{definition}

There is also a work-energy equivalent in angular momentum

\begin{equation}
	W_\mathrm{torque} = \int_{\theta_0}^\theta \tau_\mathrm{cm} \D \theta
\end{equation}