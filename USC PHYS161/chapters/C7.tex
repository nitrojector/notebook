\chapter{Gravitation}

\section{Kepler's Laws}

\begin{definition}[Kepler's Laws]
	Kepler studied and culminated his study with 3 laws:
	\begin{enumerate}
		\item Planets move in elliptical orbits with the Sun at one focus (out of two possible foci).
		\item Equal areas in equal times.
		\item The period of revolution $T$ is related to the semi-major axis $A$ by
		
		\begin{equation}
			T^2 = k A^3
		\end{equation}

		where $k$ is some constant for all planets.
	\end{enumerate}
\end{definition}

\section{Newton's Law of Universal Gravitation}

The \textbf{gravitational force} $F_g$ between two point masses $m_1$ and $m_2$ is given by

\begin{equation}
	\vec F_\mathrm{G12} = - \mathrm{G} \frac{m_1 m_2}{r^2} \hat r
\end{equation}

where $r = \abs{\vec r_1 - \vec r_2}$ is the distance between particles, and $\hat r$ is the direction of $\vec r_2 - \vec r_1$. and $\mathrm{G}$, \textbf{Newton's Constant}, is

\begin{equation}
	\mathrm{G} = \SI{6.67e-11}{N m^2 / kg^2}
\end{equation}

This is also an inverse square law

\begin{equation}
	F_\mathrm{G} \propto \frac{1}{r^2}
\end{equation}

Gravity is incredibly small compared to other fundamental forces.

Often, one mass is significantly larger than the other (i.e. Sun \& Planet) in which case a language is used: supposed $m_1 \gg m_2$, write $M \equiv m$, $m \equiv m_2$, then

\begin{equation}
	F_G = - G \frac{Mm}{r^2} \hat r
\end{equation}

is force on $m$ due to $M$.

Call $M$ the ``source'' mass and $m$ is the ``test'' mass because studying only motion of $m$.

Law is for point masses, however it holds in same form for spherical masses where $u$ is the distance between their centers.

\section{Connection to Weight / Surface Gravity}

Consider a mass $m$ at/near the surface of $M$, let the distance from the center of $m$ to the surface of $M$ be $h$, then we have

\begin{equation}
	F_\mathrm{G} = - \mathrm{G} \frac{Mm}{r^2} \hat r = - \mathrm{G} \frac{Mm}{(R + h)^2} \hat r = - \left(\frac{GM}{R^2}\right) m \frac{1}{(1 + \nicefrac{h}{R})^2}
\end{equation}

If $h \ll R$, the $\nicefrac{h}{R} \ll 1$. Let $\delta \equiv \nicefrac{h}{R}$, then we have $\frac{1}{(1 + \delta)^2}$ with $\delta \ll 1$.

Taylor expansion about $\delta = 0$ gives

\begin{equation}
	(1 + \delta)^{-2} \approx 1 - 2 \delta + O(\delta^2)
\end{equation}

Then,

\begin{equation}
	F_\mathrm{G} \approx  - \left(\frac{GM}{R^2}\right) m \left[1 - 2 \left(\frac{h}{R}\right) + O\left(\left(\frac{h}{R}\right)^2\right)\right] \hat r
\end{equation}

Since $\nicefrac{h}{R} \ll 1$, we can neglect the terms of order $\nicefrac{h}{R}$ and higher

\begin{equation}
	\vec F_\mathrm{G} \cong - \left(\frac{GM}{R^2}\right) m \hat r
\end{equation}

Consider Earth, then we have

\begin{equation}
	\frac{\mathrm{G} M_E}{{R_E}^2} = \SI{9.81}{m/s^2} \equiv \mathrm{g}
\end{equation}

Hence, $\vec F_\mathrm{G} = - mg\hat z$ is the weight.

In general, we can write gravity as the \textbf{gravitational field}

\begin{equation}
	\vec{g} (r) \equiv - \frac{\mathrm{G} M}{r^2} \hat r
\end{equation}

sourced by $M$.

\section{Principle of Equivalence}

Let's consider $m_\mathrm{G} = \nicefrac{\vec F_\mathrm{G}}{\vec g}$ as the \textbf{gravitational mass} and $m_I = \nicefrac{\vec F_\mathrm{net}}{\vec a}$ to be the inertial mass.

These have be repeatedly proven to be the same.

Thus, for a particle accelerating via gravity

\begin{equation}
	\vec a = \vec g
\end{equation}

\begin{remark}
	But, think of a situation where we are in a box, and we experience our own weight. Do we know if we are on Earth or being accelerated at gravitational acceleration on earth?

	Also, when we are experiencing free-fall, we don't experience our own weight, and there doesn't seem to be a force on us.

	We don't know. They are the same. So what Einstein figured was that gravity is not a force, but a curvature in spacetime. \footnote{If a man falls from the roof of a house, he would not feel his own weight.}
\end{remark}
