\chapter{Gravitation}

\section{Kepler's Laws}

\begin{definition}[Kepler's Laws]
	Kepler studied and culminated his study with 3 laws:
	\begin{enumerate}
		\item Planets move in elliptical orbits with the Sun at one focus (out of two possible foci).
		\item Equal areas in equal times.
		\item The period of revolution $T$ is related to the semi-major axis $A$ by
		
		\begin{equation}
			T^2 = k A^3
		\end{equation}

		where $k$ is some constant for all planets.
	\end{enumerate}
\end{definition}

\section{Newton's Law of Universal Gravitation}

The \textbf{gravitational force} $F_g$ between two point masses $m_1$ and $m_2$ is given by

\begin{equation}
	\vec F_\mathrm{G12} = - \mathrm{G} \frac{m_1 m_2}{r^2} \hat r
\end{equation}

where $r = \abs{\vec r_1 - \vec r_2}$ is the distance between particles, and $\hat r$ is the direction of $\vec r_2 - \vec r_1$. and $\mathrm{G}$, \textbf{Newton's Constant}, is

\begin{equation}
	\mathrm{G} = \SI{6.67e-11}{N m^2 / kg^2}
\end{equation}

This is also an inverse square law

\begin{equation}
	F_\mathrm{G} \propto \frac{1}{r^2}
\end{equation}

Gravity is incredibly small compared to other fundamental forces.

Often, one mass is significantly larger than the other (i.e. Sun \& Planet) in which case a language is used: supposed $m_1 \gg m_2$, write $M \equiv m$, $m \equiv m_2$, then

\begin{equation}
	F_G = - G \frac{Mm}{r^2} \hat r
\end{equation}

is force on $m$ due to $M$.

Call $M$ the ``source'' mass and $m$ is the ``test'' mass because studying only motion of $m$.

Law is for point masses, however it holds in same form for spherical masses where $u$ is the distance between their centers.

\section{Connection to Weight / Surface Gravity}

Consider a mass $m$ at/near the surface of $M$, let the distance from the center of $m$ to the surface of $M$ be $h$, then we have

\begin{equation}
	F_\mathrm{G} = - \mathrm{G} \frac{Mm}{r^2} \hat r = - \mathrm{G} \frac{Mm}{(R + h)^2} \hat r = - \left(\frac{GM}{R^2}\right) m \frac{1}{(1 + \nicefrac{h}{R})^2}
\end{equation}

If $h \ll R$, the $\nicefrac{h}{R} \ll 1$. Let $\delta \equiv \nicefrac{h}{R}$, then we have $\frac{1}{(1 + \delta)^2}$ with $\delta \ll 1$.

Taylor expansion about $\delta = 0$ gives

\begin{equation}
	(1 + \delta)^{-2} \approx 1 - 2 \delta + O(\delta^2)
\end{equation}

Then,

\begin{equation}
	F_\mathrm{G} \approx  - \left(\frac{GM}{R^2}\right) m \left[1 - 2 \left(\frac{h}{R}\right) + O\left(\left(\frac{h}{R}\right)^2\right)\right] \hat r
\end{equation}

Since $\nicefrac{h}{R} \ll 1$, we can neglect the terms of order $\nicefrac{h}{R}$ and higher

\begin{equation}
	\vec F_\mathrm{G} \cong - \left(\frac{GM}{R^2}\right) m \hat r
\end{equation}

Consider Earth, then we have

\begin{equation}
	\frac{\mathrm{G} M_E}{{R_E}^2} = \SI{9.81}{m/s^2} \equiv \mathrm{g}
\end{equation}

Hence, $\vec F_\mathrm{G} = - mg\hat z$ is the weight.

In general, we can write gravity as the \textbf{gravitational field}

\begin{equation}
	\vec{g} (r) \equiv - \frac{\mathrm{G} M}{r^2} \hat r
\end{equation}

sourced by $M$.

\section{Principle of Equivalence}

Let's consider $m_\mathrm{G} = \nicefrac{\vec F_\mathrm{G}}{\vec g}$ as the \textbf{gravitational mass} and $m_I = \nicefrac{\vec F_\mathrm{net}}{\vec a}$ to be the inertial mass.

These have be repeatedly proven to be the same.

Thus, for a particle accelerating via gravity

\begin{equation}
	\vec a = \vec g
\end{equation}

\begin{remark}
	But, think of a situation where we are in a box, and we experience our own weight. Do we know if we are on Earth or being accelerated at gravitational acceleration on earth?

	Also, when we are experiencing free-fall, we don't experience our own weight, and there doesn't seem to be a force on us.

	We don't know. They are the same. So what Einstein figured was that gravity is not a force, but a curvature in spacetime. \footnote{If a man falls from the roof of a house, he would not feel his own weight.}
\end{remark}

\section{Gravitational Potential}

Gravity is a central force and hence conservative. The gravitational potential energy $U_\mathrm{G}$ is

\begin{equation}
	U_\mathrm{G}(r) - U_\mathrm{G}(\infty) = - \int_\infty^r \vec F_\mathrm{G} \cdot \D \vec l = \int_\infty^r \frac{\mathrm{G} M m}{r^2} \D r = - \frac{\mathrm{G} M m}{r}
\end{equation}

So, we have the gravitational potential.

\begin{definition}[Gravitational Potential Energy]
	The gravitational potential energy is defined as
	
	\begin{equation}
		U_\mathrm{G}(r) = - \frac{\mathrm{G} M m}{r} 
	\end{equation}
\end{definition}

We define the gravitational potential (\textbf{not} energy) $\Phi$ by

\begin{equation}
	\Phi = \frac{U_\mathrm{G}}{m} = -\frac{\mathrm{G}M}{r}
\end{equation}

Now, if also find that

\begin{equation}
	- \nabla \Phi = - \frac{\partial \Phi}{\partial r} \hat r = - \frac{\mathrm{G}M}{r^2} \hat r = \vec g
\end{equation}

Hence, we have that

\begin{equation}
	\vec g = - \nabla \Phi
\end{equation}

and is analogous to

\begin{equation}
	\vec F_\mathrm{G} = - \nabla U_\mathrm{G}
\end{equation}

\begin{remark}
	Note that in E\&M, we have something very similar:

	\begin{equation}
		\vec F_E = - \nabla U_E \quad \vec E = \frac{\vec F_E}{q} \quad V = \frac{U_E}{q} \quad \vec E = - \nabla V
	\end{equation}
\end{remark}

\section{Two-Body Problem}

A system of two masses $m_1$ and $m_2$ , that interact solely via a central force.

\begin{align}
	m_1 \ddot{\mathbf{r}}_1 &= f(r) \hat r\\
	m_2 \ddot{\mathbf{r}}_2 &= - f(r) \hat r\\
	\mathbf{r} &\equiv \mathbf{r}_1 - \mathbf{r}_2
\end{align}

expressed explicitly

\begin{align}
	m_1 \ddot{\mathbf{r}}_1 &= - \frac{\mathrm{G}m_1 m_2}{\abs{\mathbf{r}_1 - \mathbf{r}_2}^3} (\mathbf{r}_1 - \mathbf{r}_2)\\
	m_2 \ddot{\mathbf{r}}_1 &= + \frac{\mathrm{G}m_1 m_2}{\abs{\mathbf{r}_1 - \mathbf{r}_2}^3} (\mathbf{r}_1 - \mathbf{r}_2)
\end{align}

instead, we can rewrite it as

\begin{align}
	\ddot{\mathbf{r}}_1 &= \frac{1}{m_1} f(r) \hat{\mathbf{r}}\\
	\ddot{\mathbf{r}}_1 &= - \frac{1}{m_2} f(r) \hat{\mathbf{r}}
\end{align}

we combine these equations to obtain

\begin{equation}
	\ddot{\mathbf{r}}_1 - \ddot{\mathbf{r}}_2 = \ddot{\mathbf{r}} = \left(\frac{1}{m_1} + \frac{1}{m_2}\right) f(r) \hat{\mathbf{r}}
\end{equation}

Where

\begin{equation}
	\mu \equiv \frac{m_1 m_2}{m_1 + m_2}
\end{equation}

is the reduced mass.

Our tw-body problem is equivalent to the one-body problem

\begin{equation}
	\mu \ddot{\mathbf{r}} = - f(r) \hat{\mathbf{r}}
\end{equation}

If we look at the C.O.M. frame

\begin{equation}
	\mathbf{R} = \frac{m_1 \mathbf{r}_1 + m_2 \mathbf{r}_2}{m_1 + m_2}
\end{equation}

We have

\begin{align}
	\mathbf{r}_1 &= \mathbf{r}_1' + \mathbf{R}\\
	\textbf{r}_1' &= \textbf{r}_1 - \mathbf{R}\\
	&= \frac{(m_1 + m_2) \mathbf{r}_1 - m_1 \mathbf{r}_1 - m_2 \mathbf{r}_2}{m_1 + m_2}\\
	&= \frac{m_2 (\mathbf{r}_1 - \mathbf{r}_2)}{m_1 + m_2}\\
	&= \frac{\mu}{m_1} \mathbf{r}
\end{align}

so we have

\begin{equation}
	\mu \mathbf{r} = m_1 \mathbf{r}_1'
\end{equation}

hence if we know $\mathbf{r}$ we know $\mathbf{r}_1$ and $\mathbf{r}_2$.

Now, we should consider the \textbf{conservation of angular momentum}.

\begin{align}
	\abs{\mathbf{L}} &= \abs{\mathbf{r} \cross \mu \dot{\mathbf{r}}} = \abs{\mathbf{r} \cross \mu (\dot r \hat{\mathbf{r}} + r \dot\theta \hat{\mathbf{\theta}})}\\
	&= \mu r^2 \dot \theta
\end{align}

This is basically Kepler's 2nd Law!

Conservation of Energy or the frist integral of motion:

We have total energy $E$, a constant, as:

\begin{equation}
	E = \frac{1}{2} \mu \dot{\mathbf{r}}^2 + U(r)
\end{equation}

where $f(r) \hat{\mathbf{r}} = - \nabla U$.

In polar coordinates:

\begin{align}
	\dot{\mathbf{r}} &= \dot r \hat r + r \dot\theta \hat\theta \implies \dot{\mathbf{r}}^2 = \dot r^2 + r^2 \dot \theta^2\\
	E &= \frac{1}{2} \mu \dot r^2 + \frac{1}{2}\mu r^2 \dot\theta^2 + U(r)
\end{align}

Since $L$ is constant, we write $\dot \theta = \frac{L}{\mu r^2}$ and substitute:

\begin{align}
	E &= \frac{1}{2} \mu \dot r^2 + \frac{1}{2}\mu r^2 \left(\frac{L}{\mu r^2}\right)^2 + U(r)\\
	&= \frac{1}{2} \mu \dot r^2 + U_\mathrm{eff}(r)
\end{align}

where

\begin{equation}
	U_\mathrm{eff}(r) = \frac{L^2}{2 \mu r^2} + U(r)
\end{equation}

which is the effective potential.

For gravity $U(r) = - \nicefrac{C}{r}$ where $C = \mathrm{G} m_1 m_2$

The effective potential is:

\begin{equation}
	U_\mathrm{eff}(r) = \frac{L^2}{2 \mu r^2} - \frac{C}{r}
\end{equation}

Now, we return to solving for the motion of the bodies; we want the trajectory $r = r(\theta)$.

\begin{equation}
	\frac{\D \theta}{\D t} = \frac{\D \theta}{\D r} \frac{\D r}{\D t} = \frac{\D \theta}{\D r} \dot r
\end{equation}

From energy we have

\begin{align}
	E &= \frac{1}{2} \mu \dot r^2 + \frac{L^2}{2\mu r^2} - \frac{C}{r}\\
	\dot r &= \sqrt{\frac{2}{\mu} \left(E + \frac{C}{r} - \frac{L^2}{2\mu r^2}\right)}
\end{align}

From angular momentum, we have

\begin{align}
	\dot \theta &= \frac{L}{\mu r^2}\\
	\frac{\D \theta}{\D r} &= \frac{L}{\mu r^2}\\
	\frac{\D \theta}{\D r} &= \frac{L}{\mu r^2} \frac{1}{\sqrt{\frac{2}{\mu} \left(E + \frac{C}{r} - \frac{L^2}{2\mu r^2}\right)}}\\
	&= \frac{L}{r \sqrt{2 \mu \left(Er^2 + Cr - \nicefrac{L^2}{2\mu}\right)}}\\
	\Aboxed{\frac{\D \theta}{\D r} &= \frac{L}{r \sqrt{2 \mu E r^2 + 2 \mu Cr - L^2}}}
\end{align}

Then we want to find

\begin{equation}
	\int_{\theta_0}^\theta \D \theta = \theta - \theta_0 = \int_{r_0}^r \frac{L \D r}{r \sqrt{2 \mu E r^2 + 2 \mu Cr - L^2}}
\end{equation}

The result is (process not shown):

\begin{equation}
	\theta - \theta_0 = - \arcsin \left[\left(- \frac{\mu C}{L^2}\right) \sqrt{\frac{L^4}{2\mu E L^2 + (\mu C)^2}}\right]
\end{equation}

We invert and rearrange to obtain

\begin{equation}
	r = \frac{r_0}{1 - \epsilon \cos\theta}
\end{equation}

where

\begin{align}
	r_0 &\equiv \frac{L^2}{\mu C}\\
	\epsilon &\equiv \sqrt{1 + \frac{2 E L^2}{\mu C^2}}
\end{align}

\begin{remark}
	$r_0$ is minimum of $U_\mathrm{eff}(r)$ if $E = U_\mathrm{min}$ then $\epsilon = 0 \implies r = r_0$

	If $E = U_\mathrm{min}$ the orbits are then circles.

	If $E < 0$, then $0 \leq \epsilon \leq 1$

	\medskip

	\begin{align}
		\sqrt{x^2 + y^2} &= \frac{r_0}{1 - \frac{\epsilon x}{r}}\\
		\sqrt{x^2 + y^2} \left(1 - \frac{\epsilon x}{r}\right)&= r_0\\
		\sqrt{x^2 + y^2} - \epsilon x &= r_0\\
		x^2 + y^2 &= {r_0}^2 + \epsilon^2 x^2 + 2 \epsilon r_0 x\\
		(1 - \epsilon^2)x^2 + y^2 - 2 \epsilon r_0 x &= {r_0}^2
	\end{align}

	Notice that this is the equation of an ellipse!

	$\epsilon$ is \textbf{eccentricity} of the ellipse, which is basically Kepler's 1st Law!
\end{remark}

Now, with the period of orbit

\begin{align}
	\left(\frac{\D r}{\D t}\right)^2 &= \frac{2}{\mu} (E - U_\mathrm{eff})\\
	t_a - t_b &= \int_{r_a}^{r_b} \frac{\D r}{\sqrt{\frac{2}{\mu} (E - U_\mathrm{eff})}}
\end{align}

If we compute the integral and take $r_a = r_b \implies t_a - t_b = T$ period.

\begin{align}
	T &= - \frac{\mu C \pi}{E} \frac{1}{\sqrt{-2\mu E}}\\
	T^2 &= \frac{\mu \pi^2}{2 C} \left(- \frac{C^3}{E^3}\right)
\end{align}

Since $-1 < \cos\theta < 1 \implies r_\mathrm{min} < r < r_\mathrm{max}$ where $r_\mathrm{max} = \frac{r_0}{1 - \epsilon}$ and $r_\mathrm{min} = \frac{r_0}{1 + \epsilon}$

\begin{equation}
	T^2 = \frac{\pi^2}{2G(m_s + m_p)} A^3
\end{equation}

%! Possible more to add