\chapter{Kinematics}

We have our position vector

\begin{equation}
	\vec r(t) = (x(t), y(t))
\end{equation}

We use $\vec r$ because it seems natural, it is the direction we are pointing in.

\begin{remark}
	Sometimes when reference to radial $\vec r$ is misleading, we use $\vec x(t)$.
\end{remark}

The change of the vector in space across time sweeps over some \textbf{trajectory}.

\section{Displacement}

\begin{definition}[Displacement]
	The \textit{displacement vector} $\Delta\vec r$ is a measure of where the particle went (which depends on the origin!).

	\begin{equation}
		\Delta\vec r \equiv \vec r_f - \vec r_i = \vec r(t_f) - \vec r(t_i)
	\end{equation}
\end{definition}

\begin{enumerate}
	\item $\norm{\Delta \vec r} \neq$ distance travelled in general
	\begin{itemize}
		\item distance traveled = arc length of trajectory
	\end{itemize}

	\item $\Delta \vec r$ is coordinate independent.
	
	Take two coordinate systems $S$ and $S'$. Let them be defined with the relation $\vec r = \vec r' + \vec R$ where $\vec r$ and $\vec r'$ are vectors in the respective coordinate systems.

	\begin{equation}
		\begin{cases}
			S: &\Delta \vec r = \vec r_f - \vec r_i\\
			S':&\Delta \vec r' = \vec {r_f}' - \vec {r_i}'
		\end{cases}
	\end{equation}

	If we plug in the relation, we realize that they are the same, $\Delta \vec r = \Delta \vec r'$
\end{enumerate}

\section{Velocity}

\begin{definition}[Average Velocity]
	\begin{equation}
		\vec v_{\mathrm{avg}} \equiv \frac{\Delta \vec r}{\Delta t}
	\end{equation}
\end{definition}

Let $\D \vec r$ be the infinitesimal displacement.

When we consider a smaller interval:

\begin{equation}
	\lim_{\Delta t \to 0} \implies \norm{\D \vec r} = \D r \quad \text{(distance traveled)}
\end{equation}

A small change to $t$ results in a small change in $\D S$ (the distance / speed), proportionally

\begin{align}
	&\D S \propto \D t\\
	\implies &\D S = \left(\frac{\D S}{\D t}\right) \D t
\end{align}

\begin{definition}[Velocity]
	AKA the \textit{instantaneous velocity}

	\begin{equation}
		\vec v (t) \equiv \frac{\D \vec r}{\D t}
	\end{equation}

	\begin{itemize}
		\item $\norm{\vec v}$ = speed
		\item $\hat v$ = direction of motion
	\end{itemize}
\end{definition}

\begin{remark}
	A note on average velocity:

	\begin{equation}
		\vec v_{\mathrm{avg}} = \frac{1}{\Delta t} \int_{t_i}^{t_f} \vec v(t) \D t = \frac{1}{\Delta t} \int_{t_i}^{t_f} \frac{\D \vec r}{\D t} \D t = \frac{\Delta \vec r}{\Delta t}
	\end{equation}

	Note also if we find the magnitude, it would not be the same as the average speed since the norm would go over the integrals instead of what is being integrated.
\end{remark}

\begin{itemize}
	\item $\vec v$ a vector, so write $\vec v(t) = \dot x \hat x + \dot y \hat y = \dot{\vec{r}}$
	\item Compare to frames of reference, $S$ \& $S'$
	
	Suppose $\dot{\vec{R}} \neq 0$.

	Then we have

	\begin{equation}
		\begin{cases}
			\vec r &= \vec r' + \vec R\\
			\vec v &= \vec v' + \vec V
		\end{cases}
	\end{equation}

	This is known as the Galilean transformations, which, at higher velocities, ``translates'' to the Lorentz transformations.
\end{itemize}

We can also obtain $\vec r(t)$ given $\vec v(t)$

\begin{equation}
	\Delta \vec r = \int \D \vec r = \int_{t_i}^{t_f} \vec v \D t
\end{equation}

and

\begin{align}
	\vec r(t) &= \vec r_i + \vec v_i (t - t_i)
\end{align}

\section{Acceleration}

\begin{definition}
	\begin{equation}
		\vec a(t) = \frac{\D \vec v}{\D t} = \frac{\D^2 \vec r}{\D^2 t}
	\end{equation}
\end{definition}

Similar to what is mentioned in \cref*{sect:calc-with-vec}, $\vec a_\parallel$ is change in speed, $\vec a_\perp$ is change in direction of motion.

\begin{remark}
	We do have the \textit{jerk}, but it just seems that it never really matters, and acceleration is fully sufficient.
\end{remark}

\subsection{Cartesian Coordinates}

\begin{equation}
	\begin{cases}
		\vec r(t) &= x(t)\hat x + y(t)\hat y + z(t)\hat z\\
		\vec v(t) &= \dot x(t)\hat x + \dot y(t)\hat y + \dot z(t)\hat z\\
		\vec a(t) &= \ddot x(t)\hat x + \ddot y(t)\hat y + \ddot z(t)\hat z
	\end{cases}
\end{equation}

\begin{example}
	Suppose particle's position is $\vec r(t) = A(e^{\alpha t} \hat x + e^{-\alpha t} \hat y)$ with $A$ and $\alpha$ constants. ($[A] = \mathrm{m}, [\alpha] = \mathrm{m}^{-1}$) Find $\vec v(t)$ and $\vec a(t)$ and sketch trajectory.
\end{example}

\begin{sol}
	\textbf{Velocity}

	\begin{align}
		\vec v(t) &= \frac{\D \vec r}{\D t}\\
		&= A(\alpha e^{\alpha t} \hat x - \alpha e^{-\alpha t} \hat y)\\
		&= \alpha A(e^{\alpha t} \hat x - e^{-\alpha t} \hat y)
	\end{align}

	\textbf{Acceleration}

	\begin{align}
		\vec a(t) &= \frac{\D \vec v}{\D t}\\
		&= \alpha^2 A(e^{\alpha t} \hat x + e^{-\alpha t} \hat y)\\
		&= \alpha^2 \vec r(t)
	\end{align}

	\textbf{Speed}

	\begin{align}
		\abs{\vec v} &= \sqrt{\vec v \cdot \vec v}\\
		&= \sqrt{(\alpha A)^2 \left[e^{2\alpha t} + e^{-2\alpha t}\right]}\\
		&= \alpha A \sqrt{2\cosh(2\alpha t)}
	\end{align}

	Note that (by definition)

	\begin{equation}
		\begin{cases}
			x(t) &= Ae^{\alpha t}\\
			y(t) &= Ae^{-\alpha t}
		\end{cases}
	\end{equation}

	We can try to find $y(x)$ by eliminating $t$, which is the equation for the trajectory, we obtain:

	\begin{equation}
		y(x) = \frac{A^2}{x} \qquad y \propto \frac{1}{x}
	\end{equation}

	So although the velocity and acceleration changes at a exponential rate, the trajectory that it produces exhibits the inverse curve.
\end{sol}

\begin{example}
	A particle moves in the plane with trajectory of a circle of radius $R$. The particle sweeps out the circle at a uniform and constant rate. That is, it undergoes uniform circular motion. Find $\vec r(t), \vec v(t),$ and $\vec a(t)$.
\end{example}

\begin{sol}
	% diagram

	We know that the magnitude of the position vector $\abs{\vec r} = R$ and that

	\begin{equation}
		\vec r = R\cos\theta(t) \hat x + R\sin\theta(t) \hat y
	\end{equation}

	\begin{remark}
		$\vec v$ changes direction, but with uniform rate $\abs{\vec v} = c$.
	\end{remark}

	From our $\vec r(t)$ we have that

	\begin{align}
		\vec v(t) &= -R\sin\theta(t) \left(\frac{\D \theta}{\D t}\right) \hat x + R\cos\theta(t) \left(\frac{\D \theta}{\D t}\right) \hat y\\
		&= R\dot\theta \left[-\sin\theta \hat x + \cos\theta \hat y\right]
	\end{align}

	We know that $v$ is constant and that $v = R\dot\theta$, so $R\dot\theta$ must also be constant. Since $R$ is constant, $\dot\theta$ is constant.

	\begin{equation}
		\dot\theta \equiv \omega \implies \theta(t) = \omega t
	\end{equation}

	This is assuming $\theta(0) = 0$.

	As a result of our derivation, we find

	\begin{equation}
		\begin{cases}
			\vec r(t) &= R\cos(\omega t)\hat x + R\sin(\omega t) \hat y\\
			\vec v(t) &= -\omega R\sin(\omega t) \hat x + \omega R\cos(\omega t) \hat y
		\end{cases}
	\end{equation}

	Now, noting the magnitude:

	\begin{equation}
		\begin{cases}
			r &= R\\
			v &= \omega R\\
			a &= \omega^2 R = \frac{v^2}{R}
		\end{cases}
	\end{equation}

	\textbf{Acceleration}

	\begin{align}
		\vec a(t) &= -\omega^2 R \cos(\omega t) \hat x - \omega^2 R \sin(\omega t) \hat y\\
		&= -\omega^2 \vec r(t)
	\end{align}

	\begin{remark}
		Because $\hat a$ points towards the origin [$\hat a = - \hat r$], we call it ``centripetal'' ($\leftarrow$ central seeking).
	\end{remark}
\end{sol}

\section{Formal Solution of Kinematic Equations}

We want to obtain $\vec v(t)$ and $\vec r(t)$ given $\vec a(t)$.

\subsection{$\vec v$ from $\vec a$}

\begin{align}
	\int_0^t \vec a(t) \D t' &= \int_{\vec v_0}^{\vec v} \frac{\D \vec v}{\D t'} \D t'\\
	&= \vec v(t) - \vec v_0\\
	\vec v(t) &= \boxed{\vec v_0 + \int_0^t \vec a(t') \D t'}
\end{align}

\subsection{$\vec r$ from $\vec v$ (from $\vec a$)}

\begin{align}
	\int_0^t \vec v(t') \D t' &= \int_{\vec r_0}^{\vec r} \frac{\D \vec r}{\D t} \D t\\
	&= \vec r(t) - \vec r_0\\
	\vec r(t) &= \boxed{\vec r_0 + \int_0^t \vec v(t') \D t'}\\
	&= \vec r_0 + \int_0^t \left[\vec v_0 + \int_0^{t'} \vec a(t'') \D t''\right] \D t'\\
	\vec r(t) &= \boxed{\vec r_0 + \vec v_0 t + \int_0^t \int_0^{t'} \vec a(t'') \D t'' \D t'}
\end{align}

\begin{remark}
	We need to know $\vec r_0$.

	To find $\vec r(t)$ given $\vec a(t)$ we need also to know the initial conditions, $\vec r_0$ and $\vec v_0$.
\end{remark}

\section{Constant Acceleration Motion}

\begin{theorem}[Kinematic Equations with Constant $\vec a$]
	There are many cases of constant $\vec a$ motion. With our previous analysis, the cases of when $\vec a =$ const gives:

	\begin{equation}
		\begin{cases}
			\vec r(t) &= \vec r_0 + \vec v_0 t + \frac{1}{2} \vec a t^2\\
			\vec v(t) &= \vec v_0 + \int_0^t \vec a \D t' = \vec v_0 + \vec a t\\
			v^2 &= {v_0}^2 + 2 \vec a \cdot \Delta \vec r
		\end{cases}
	\end{equation}
\end{theorem}

\begin{remark}
	if $t_0 \neq 0$, the $t \to \Delta t$ in formulas.
\end{remark}

Let's eliminate $t$ from these equations:

From $\vec v = \vec v_0 + \vec a t$, compute $v^2 = \vec v \cdot \vec v$

\begin{align}
	v^2 &= {v_0}^2 + 2 \vec v_0 \cdot \vec a t + a^2t^2\\
	\frac{1}{2} v^2 &= \frac{1}{2} {v_0}^2 + \vec v_0 \cdot \vec a t + \frac{1}{2}a^2t^2
\end{align}

Now, from $\vec r$ compute

\begin{equation}
	\vec a \cdot \vec r = \vec a \cdot \vec r_0 + \vec a \cdot \vec v_0 t + \frac{1}{2} a^2 t^2
\end{equation}

Then, we take the difference, we have

\begin{align}
	\frac{1}{2} v^2 - \vec a \cdot \vec r &= \frac{1}{2} {v_0}^2 - \vec a \cdot \vec r_0\\
	\frac{1}{2} v^2 &= \frac{1}{2} {v_0}^2 + \vec a \cdot (\vec r - \vec r_0)\\
	\Aboxed{v^2 &= {v_0}^2 + 2 \vec a \cdot \Delta \vec r}
\end{align}

\subsection{Components of the Equations}

\begin{remark}
	These laws are also applicable in components.
\end{remark}

% !TBA

\section{Two-Dimensional Motion}

\subsection{Free Fall}

All objects regardless of mass, shape, composition, etc., fall downward towards earth with same motion -- \textit{free fall}.

Free fall is vertical motion subject \textit{only} to earth's gravity, which is constant acceleration motion.

The acceleration due to gravity, $g$, is 

\begin{equation}
	g = \SI{9.8}{\m\per\s^2} \qquad \vec a = -g \hat z\footnotemark
\end{equation} \footnotetext{True near earth's surface}

\subsection{Projectile Motion}

\textit{Projectile motion} is motion subject only to gravity, that is, motion for which $\vec a = -g \hat z$.

\begin{remark}
	Projectile motion lies in the plane formed by $\vec v_0$ and $\vec a$. This implies 2D motion.
\end{remark}

Now, the equations:

% \begin{align}
% 	\vec r(t) &= \vec r_0 + \vec v_0 t - \frac{1}{2} (\g \hat z) t^2\footnotemark\\
% 	\vec v(t) &= \vec v_0 - (g \hat z) t\\
% 	v^2 &= {v_0}^2 + 2 \vec (g \hat z) \cdot \Delta \vec r
% \end{align}
% \footnotetext[2]{We substitute the acceleration of gravity}

But a lot of times what we do is to consider the two components in Cartesian.

\begin{align}
	\text{$x$ component} &\implies x(t) = x_0 + v_{0x}t\\
	\text{$y$ component} &\implies 0\\
	\text{$z$ component} &\implies \begin{cases}
		z(t) &= z_0 + v_{0z} t - \frac{1}{2}gt^2\\
		v_z(t) &= v_{0z} - gt\\
		{v_z}^2 &= {v_{0z}}^2 - 2g\Delta z
	\end{cases}
\end{align}

\begin{example}
	Consider a projectile launched with initial velocity $\vec v_0$ that makes angle $\theta$ with the horizontal. Choose coordinates s.t. $(x_0, y_0, z_0) = (0,0,h)$ with the plane of motion the $xz$-plane.

	Find:
	\begin{enumerate}[a)]
		\item the trajectory of the projectile, $z = z(x)$
		\item the maximum height and horizontal distance (i.e. range) of the projectile
		\item the velocity of the projectile when it hits the ground
		\item the launch angle, $\phi$, that maximizes the range. Here, let $h = 0$.
	\end{enumerate}
\end{example}

\begin{sol}
	\begin{enumerate}[a)]
		\item Equations for the motion are:
		
		\begin{equation}
			\begin{cases}
				z(t) &= h + v_0 \sin\theta t + \frac{1}{2} g t^2\\
				v_z(t) &= v_0 \sin\theta + gt\\
				{v_z}^2 &= {v_0}^2 \sin^2\theta - 2g(z-h)\\
				x(t) &= v_0 \cos\theta t
			\end{cases}
		\end{equation}

		We simply have to find $z$ in terms of $x$, notice how $z(x) = z(t(x))$. We just need $t(x)$.

		We find that 

		\begin{align}
			x &= v_0\cos\theta t\\
			t &= \frac{x}{v_0\cos\theta}
		\end{align}

		Now we substitute

		\begin{align}
			z(t) &= h + v_0 \sin\theta t + \frac{1}{2} g t^2\\
			z(t) &= h + v_0 \sin\theta \left(\frac{x}{v_0\cos\theta}\right) + \frac{1}{2} g \left(\frac{x}{v_0\cos\theta}\right)^2\\
			&= \boxed{h + x\tan\theta + \frac{gx^2}{2{v_0}^2\cos^2\theta}}
		\end{align}

		\item \textbf{Maximum Heigh $z_{\mathrm{max}}$}
		
		Obtained when $v_z = 0$
		
		\begin{align}
			\implies 0 &= v_0 \sin\theta - gt_{\mathrm{max}}\\
			t_{\mathrm{max}} &= \frac{v_0 \sin\theta}{g}
		\end{align}

		Sub into $z$-equation

		\begin{align}
			z_{\mathrm{max}} &= h + v_0\sin\theta\left(\frac{v_0\sin\theta}{g}\right) - \frac{1}{2} g \left(\frac{v_0\sin\theta}{g}\right)^2\\
			&= \boxed{h + \frac{{v_0}^2\sin^2\theta}{2g}}
		\end{align}

		Alternatively,

		\begin{align}
			0 &= {v_0}^2\sin^2\theta - 2g(z_{\mathrm{max}} - h)\\
			z_{\mathrm{max}} &= \boxed{h + \frac{{v_0}^2\sin^2\theta}{2g}}
		\end{align}

		\textbf{Range $x_{\mathrm{max}}$}

		Occurs when $z = 0$

		\begin{align}
			0 &= h + v_0\sin\theta t_f - \frac{1}{2}g{t_f}^2\\
			t_f &= \frac{-v_0\sin\theta \pm \sqrt{{v_0}^2\sin^2\theta + 2gh}}{-g}\\
			&= \frac{v_0\sin\theta}{g} \mp \sqrt{\left(\frac{v_0\sin\theta}{g}\right)^2 + \frac{2h}{g}}
		\end{align}

		\begin{remark}
			We have to chose the positive of the $\mp$ because larger time.
		\end{remark}

		We notice that if $h = 0$ (more generally, $\Delta z = z_f - z_0 = 0$)

		\begin{equation}
			t_f = \frac{2v_0\sin\theta}{g} = 2t_\mathrm{max} \implies \text{symmetry of $z(t)$ parabola}
		\end{equation}

		From $x$-equation:

		\begin{equation}
			x_\mathrm{max} = \frac{{v_0}^2\sin\theta\cos\theta}{g} + v_0 \cos\theta \sqrt{\left(\frac{v_0\sin\theta}{g}\right)^2 + \frac{2h}{g}}
		\end{equation}

		Use the identity $2\sin\theta\cos\theta = \sin(2\theta)$

		Which gives us

		\begin{align}
			x_\mathrm{max} &= \boxed{\frac{{v_0}^2\sin(2\theta)}{2g} + \sqrt{\left(\frac{{v_0}^2\sin(2\theta)}{2g}\right)^2 + \frac{2h{v_0}^2\cos^2\theta}{g}}}\\
			&= \frac{{v_0}^2\sin(2\theta)}{2g} \left[1 + \sqrt{1 + \frac{2gh}{{v_0}^2\sin^2\theta}}\right]
		\end{align}

		Now, if we solve for the case wehre $h = 0$, we get

		\begin{equation}
			x_\mathrm{max} = \frac{{v_0}^2 \sin(2\theta)}{g}
		\end{equation}

		\item We want $\vec v_f$, which is $\vec v_f = \vec v_0 - gt_f \hat z$
		
		\begin{align}
			\vec v_f &= v_0\cos\theta\hat x - \left(v_0\sin\theta \sqrt{1 + \frac{2gh}{{v_0}^2\sin^2\theta}}\right) \hat z
		\end{align}

		We can also write it in terms of magnitude and angle:

		First to find the magnitude

		\begin{align}
			{v_f}^2 &= {v_0}^2\cos^2\theta + {v_0}^2\sin^2\theta \left(1 + \frac{2gh}{{v_0}^2\sin^2\theta}\right)\\
			&= {v_0}^2 + 2gh\\
			\implies v_f &= \sqrt{{v_0}^2 + 2gh}
		\end{align}

		Now, for the angle of the projectile when it hits the ground

		\begin{align}
			\tan\theta_f &= \abs{\frac{v_{fz}}{v_{fx}}} = \tan\theta\sqrt{1 + \frac{2gh}{{v_0}^2\sin^2\theta}}\\
			\theta_f &= \arctan\left[\tan\theta \sqrt{1 + \frac{2gh}{{v_0}^2\sin^2\theta}}\right]
		\end{align}

		\begin{remark}
			Notice now when $h = 0$, $\theta_f = \theta$.
		\end{remark}

		\item Since $h = 0$, the range is $x_\mathrm{max} = \frac{{v_0}^2\sin(2\theta)}{g}$
		
		We want to maximize, so we can think that $x_\mathrm{max} = x_\mathrm{max}(\theta)$ and find $\theta = \phi$ s.t. $\left.\frac{\D x_\mathrm{max}}{\D \theta} \right|_\phi = 0$

		\begin{align}
			\left.\frac{2{v_0}^2 \cos(2\theta)}{g}\right|_\phi = \frac{2{v_0}^2}{g}\cos(2\phi) &= 0\\
			\implies \cos(2\phi) &= 0\\
			\phi &= \frac{\pi}{4} = 45\deg
		\end{align}
	\end{enumerate}
\end{sol}

\begin{example}
	A hunter is trying to hunt a bear on a tree with heigh $h$ distance $d$ away. The moment the hunter shoots, the bear is scared and drops from the tree. What angle relative to the bear should the hunter aim at to hit the bear?
\end{example}

\begin{sol}
	We can consider the vertical component, which must match for the hunter's arrow to hit

	\begin{align}
		y+0 + v_{0_y} t - \frac{1}{2}gt^2&\\
		v_0\sin\theta t - \frac{1}{2}gt^2 &= h - \frac{1}{2}gt^2\\
		v_0\sin\theta t &= h\\
		t &= \frac{h}{v_0\sin\theta}
	\end{align}

	then we plug the vertical to horizontal

	\begin{align}
		\frac{h}{v_0\sin\theta}\cos\theta &= d\\
		d&= h\cot\theta\\
		\theta &= \boxed{\arccotangent\left(\frac{d}{h}\right)}
	\end{align}

	We notice that $\theta$ then is essentially directly at the bear.

	\begin{remark}
		Another way of thinking about it, is if we consider $g = 0$, then consider the problem, we would come to the conclusion that we should aim at the bear too. Adding g to both bodies shouldn't change that fact.
	\end{remark}

	Since we also want the hunder to hit the bear before it hits the ground, we can find that

	\begin{align}
		h &= \frac{1}{2}gt^2\\
		t &= \sqrt{\frac{2h}{g}}\\
		t &< \sqrt{\frac{2h}{g}}
	\end{align}

	Consequently

	\begin{align}
		\sqrt{\frac{2h}{g}} v_0 \cos\theta &= \sqrt{\frac{2h}{g}} \frac{d}{d^2 + h^2} v_0\\
		v_0 &> \boxed{\sqrt{\frac{g(d^2 + h^2)}{2d}}}
	\end{align}
\end{sol}

\begin{remark}
	The \textbf{Frenet-Serret Formulas} gives a way of finding motion only based on the particle's current motion relative to itself.
\end{remark}

\section{Kinematics in Plane Polar Coordinates}

\begin{definition}
	\begin{align}
		\vec r(t) &= r \hat r = r(t) \hat r(t)\\
		\dot{\vec r}(t) &= \dot r \hat r + r \dot{\hat r}\\
		&= \dot r \hat r + r \dot \theta \hat \theta\\
		&= \dot r \hat r + r \omega \hat \theta\\
		\vec a &= \frac{\D \vec v}{\D t} = \ddot r \hat r + \dot r \dot{\hat r} + \dot r \dot \theta \hat \theta + r \ddot \theta \hat \theta + r \dot\theta \dot{\hat \theta}\\
		&= (\ddot r - r \dot \theta^2) \hat r + (r \ddot \theta + 2 \dot r \dot \theta) \hat \theta
	\end{align}
	
	\begin{remark}
		If we have the trajectory being a circle, we have velocity $\vec v = r\omega\hat\theta$.
	\end{remark}

	$\ddot r$ is radial acceleration, and $-r\dot\theta^2 = -r\omega^2$ is the centripetal acceleration.

	$r\ddot\theta$ is angular acceleration ($\alpha \equiv \ddot\theta, r\alpha$), and $2\dot r\dot\theta$ is the Coriolis Acceleration\footnote{This is related to the Coriolis Effect in non-inertial frames.}
\end{definition}

Once again, we can see for circular motion

\begin{align}
	\dot r &= \ddot r = 0\\
	\implies \vec a &= (-r\dot\theta^2) \hat r + (r\ddot\theta)	\hat \theta = -r\omega^2\hat r + r\alpha\hat\theta
\end{align}

Note the above is from constant speed only

\begin{remark}
	Three examples from the notes are not included.
\end{remark}