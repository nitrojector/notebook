\chapter{Momentum}

\section{Introduction}

Recall that we mentioned that what Newton actually defined for the 2nd law is:

\begin{equation}
	\vec F = \frac{\D \vec p}{\D t} \qquad \boxed{\vec p = m \vec v}
\end{equation}

Consider a system of $N$ particles, unconstrained in their motion (i.e. do not necessarily have kinematic constraints)

For the $j$-th particle, the 2nd law says:

\begin{equation}
	\vec F_j  = \frac{\D \vec v_j}{\D t} \left(= m_j \frac{\D \vec v_j}{\D t}\right)
\end{equation}

By superposition, 

\begin{equation}
	\vec F_j = \sum_{i\neq j}^N \vec F_{ij} + {\vec F_j}^\mathrm{ext}
\end{equation}

where $\vec F_{ij}$ is the force of the $i$-th particle (in system) on $j$-th particle, and ${\vec F_j}^\mathrm{ext}$ is the (net) external (i.e. not in system) of $j$-th particle.

Consider the total force of the system $\sum_{j=1}^{N} \vec F_j$, then

\begin{equation}
	\sum_j \vec F_j = \sum_j \sum_{i \neq j} \vec F_{ij} + \sum_j {\vec F_j}^\mathrm{ext} \equiv \sum_j \sum_{i\neq j} \vec F_{ij} + {\vec F}^\mathrm{ext}
\end{equation}

where ${\vec F}^\mathrm{ext}$ is the total external force on system.

If we look at first terms

\begin{equation}
	\sum_j \sum_{i\neq j} \vec F_{ij} = \vec F_{12} + \vec F_{13} + \cdots + \vec F_{21} + \vec F_{23} + \cdots
\end{equation}

Note that Newton's 3rd Law $\vec F_{ij} = - \vec F_{ji}$, so all the terms in the sum cancels out.

\begin{remark}
	Since $\vec F_{ij}$ is a antisymmetric quantity, the sum over it is zero.
\end{remark}

Hence, 

\begin{equation}
	\vec F_\mathrm{tot} = \sum_j \vec F_j = \vec F^\mathrm{ext}
\end{equation}

We apply Newton's 2nd law:

\begin{equation}
	\vec F^\mathrm{ext} = \sum_j \frac{\D \vec p_j}{\D t} = \frac{\D}{\D t} \left(\sum_j \vec p_j\right)
\end{equation}

So we define the total momentum $\vec p_\mathrm{tot}$ to be

\begin{equation}
	\vec p_\mathrm{tot} \equiv \sum_{j = 1}^N \vec p_j
\end{equation}

which means that

\begin{equation}
	\vec F^\mathrm{ext} = \frac{\D \vec p_\mathrm{tot}}{\D t}
\end{equation}

\section{Center of Mass (COM)}

\subsection{Discrete Masses}

The total external force on the system changes the total momentum of the system as if it were a point particle, let's try to write that in the form $\vec F^\mathrm{ext} = M \vec A$.

\begin{align}
	\vec F^\mathrm{ext} &= \frac{\D \vec p_\mathrm{tot}}{\D t} = \frac{\D}{\D t} \left(\sum_j m_j \vec v_j\right) = \sum_j m_j \frac{\D \vec v_j}{\D t}\\
	&= \sum_j m_j \frac{\D^2 \vec r_j}{\D t^2} = \frac{\D^2}{\D t^2} \sum_j m_j \vec r_j\\
	&= \left(\frac{\sum m_j}{\sum m_j}\right) \frac{\D^2}{\D t^2} \sum_j m_j \vec r_j
\end{align}

So we define the \textit{total mass} $M$ by

\begin{equation}
	M \equiv \sum_j m_j
\end{equation}

and define the (position of) \textit{center of mass}, $\vec R$ by

\begin{equation}
	\vec R \equiv \frac{1}{M} \sum_j m_j \vec r_j
\end{equation}

THen,

\begin{equation}
	\vec F = M \frac{\D^2 \vec R}{\D t^2}
\end{equation}

or if $\vec V = \frac{\D \vec R}{\D t}$ and $\vec A = \frac{\D \vec V}{\D t}$, that means $\vec F^\mathrm{ext} = M \vec A$, and so there we have it.

\begin{example}
	Say a combined particle of A ($m_A = \frac{1}{4} M$) and B ($m_B = \frac{3}{4} M$) is launched at some angle $\theta$ with initial velocity $v_0$ under the influence of gravity. At the apex of the trajectory, an internal explosion occurs, repelling the two particles in opposite direction horizontally.

	Relative to the launch position, if $x_{B} = d$, find $x_A$.
\end{example}

\begin{sol}
	Let's use $\vec F^\mathrm{ext} = M \frac{\D^2 \vec R}{\D t^2}$

	we have center of mass

	\begin{equation}
		\begin{cases}
			R_x &= \frac{1}{M} (m_A x_A + m_B x_B)\\
			R_y &= \frac{1}{M} (m_A y_A + m_B y_B)\\ 
		\end{cases}
	\end{equation}

	And $\vec F^\mathrm{ext} = - Mg\hat y$

	Looking at the 2nd law of the COM:

	\begin{equation}
		\ddot R_x = 0 \qquad \ddot R_y = -g
	\end{equation}

	\begin{equation}
		\begin{cases}
			0 &= \frac{1}{M} (m_A \ddot x_A + m_B \ddot x_B)\\
			-g &= \frac{1}{M} (m_A \ddot y_A + m_B \ddot y_B)\\ 
		\end{cases}
	\end{equation}

	Our solution for COM is:

	\begin{equation}
		\vec R(t) = \vec R_0 + \vec V_0 t - \frac{1}{2} \vec g t^2
	\end{equation}

	\begin{equation}
		\begin{cases}
			R_x &= v_0 \cos \theta t\\
			R_Y &= v_0 \sin\theta t - \frac{1}{2} gt^2
		\end{cases}
	\end{equation}

	Particles hit the ground at

	\begin{equation}
		t_f = \frac{2v_0\sin\theta}{g}
	\end{equation}

	Hence

	\begin{equation}
		R_{xf} = v_0 \cos\theta \left(\frac{2v_0 \sin\theta}{g}\right) = \frac{v_0^2 \sin(2\theta)}{g}
	\end{equation}

	Consequently:

	\begin{align}
		R_{xf} &= \frac{1}{M} (m_A x_{Af} + m_B x_{Bf})\\
		\frac{v_0^2 \sin(2\theta)}{g} &= \frac{1}{M} (\frac{1}{4} M x_{Af} + \frac{3}{4} M d) = \frac{1}{4} x_{Af} + \frac{3}{4} d\\
		x_{Af} &= \boxed{\frac{4v_0^2 \sin(2\theta)}{g} - 3d}
	\end{align}
\end{sol}

\subsection{Continuous Distribution}

For a body that is (to good approximation) continuous, i.e. its mass is continuously distributed throughout its volume $V$, its center of mass becomes

\begin{equation}\label{eq:com-cont}
	\vec R = \frac{1}{M} \int_V \vec r \D m
\end{equation}

where $M = \int \D m$

Now we want to practically evaluate \cref*{eq:com-cont}, we can replace $\D m$ by:

\begin{equation}
	\D m = \begin{cases}
		\lambda(\vec r) \D l &\implies \text{1-dim distribution [mass/length]}\\
		\sigma(\vec r) \D \partial &\implies \text{2-dim distribution [mass/area]}\\
		\rho(\vec r) \D V &\implies \text{3-dim distribution [mass/vol]}
	\end{cases}
\end{equation}

\begin{remark}
	What if we have a point mass in 1-dim?

	\begin{equation}
		\int_V \D m = M
	\end{equation}

	There's the \href{https://en.wikipedia.org/wiki/Dirac_delta_function}{\textbf{Dirac Delta}} $\delta (\vec r)$ s.t.

	\begin{equation}
		\int_{\mathbb{R}^3} \delta^3(\vec r) \D^3 \vec r = 1
	\end{equation}

	So, for a point pass

	\begin{equation}
		m = m \int_{-\infty}^\infty \delta (x-a) \D x
	\end{equation}
\end{remark}

\begin{example}
	Take some example 2D object with density $\sigma$ that is a triangle of height $h$ and length $b$ with the right angle side away from the origin.
\end{example}

\begin{sol}
	\begin{align}
		\vec R &= \frac{1}{M} \int \vec r \D m\\
		&= \frac{1}{M} \vec r \sigma \D A\\
		&= \frac{\sigma}{M} \int \vec r \D A
	\end{align}

	Area is $\frac{1}{2}bh$ so $\sigma = \frac{2M}{bh}$

	\begin{align}
		\vec R &= \frac{2}{bh} \int \vec r \D A\\
		&= \frac{2}{bh} \left[\hat x \iint x \D x \D y + \hat y \iint y \D x \D y\right]\\
		&= \frac{2}{bh} \left[\hat x \int_0^b\int_0^{hx/b} x \D x \D y + \hat y \int_0^b\int_0^{hx/b} y \D x \D y\right]\\
		&= \frac{2}{3} b \hat x + \frac{1}{3} h \hat y
	\end{align}
\end{sol}

\subsection{Center of Mass Frame}

Given two particles $m_1$ and $m_2$, the center of mass is given by

\begin{equation}
	\vec R = \frac{m_1 \vec r_1 + m_2 \vec r_2}{m_1 + m_2}
\end{equation}

The center of mass frame is the frame whose origin is $\vec R$.

\begin{equation}
	\begin{cases}
		\vec r_{1_\mathrm{COM}} &= \vec r_1 - \vec R\\
		\vec r_{2_\mathrm{COM}} &= \vec r_2 - \vec R
	\end{cases}
\end{equation}

If $\ddot{\vec{R}} = 0$ that means that the frame is inertial (which is to also say that $\vec F^\mathrm{ext} = 0$)

Note also that the \textbf{momentum in the COM frame is always 0!} It ends up much easier to analyze collisions in the COM frame.

\section{Variable Mass Situations}

\begin{example}
	Let the speed oft he exhaust be $u$ relative to the rocket -- this is an inertial frame. The rocket has mass $M_R$. The fuel is ejected at a rate $\D M_F$ for small interval $\D t$.
\end{example}

\begin{sol}
	Note that $\D M_F = - \D M_R$ where $M_R$ is the mass of the rocket.

	This results in increase in rocket velocity: $v \rightarrow v + \D v$

	Assume that $u$ is constant.

	The initial momentum is then $P_0 = M_R v$.

	Final momentum, which is 

	\begin{equation}
		P_f = (M_R - \D M_F) (v + \D v) + \D M_F (v - u)
	\end{equation}

	\begin{align}
		\Delta P &= P_f - P_0\\
		&= M_R v + M_R \D v - V \D M_F - \D M_F \D v\footnotemark + V \D M_F - u \D M_F - M_R v\\
		&= M_R \D v - u \D M_F\\
		F &= M_R \frac{\D v}{\D t} - u \frac{\D M_F}{\D t}\\
	\end{align}

	\footnotetext{Note that $\D M_F \D v$ we are considering as negligible since the two differentials are both rather small}

	which, given out initial inversion statement

	\begin{equation}
		F = M_R \frac{\D v}{\D t} + u \frac{\D M_R}{\D t}
	\end{equation}
\end{sol}

\begin{example}
	A rocket of mass $m_0$ at $t = 0$, exhausts mass backward, accelerating the rocket forward in free empty space. If exhaust velocity $u$ is constant, determine the rocket's velocity as function of it's mass.
\end{example}

\begin{sol}
	We have that $F = 0$, and therefore

	\begin{equation}
		M_R \frac{\D v}{\D t} - u \frac{\D M_F}{\D t} = 0 \longrightarrow  m \frac{\D v}{\D t} + u \frac{\D m}{\D t} = 0
	\end{equation}

	Then, we can solve

	\begin{align}
		m \D v &= - u \D m\\
		\frac{\D v}{u} &= -\frac{\D m}{m}\\
		\int_0^v \frac{\D v}{u} &= - \int_{m_0}^{m} \frac{\D m}{m}\\
		\frac{v}{u} &= - \log\left(\nicefrac{m}{m_0}\right)\\
		v(m) &= \boxed{- u \log\left(\nicefrac{m}{m_0}\right)}
	\end{align}
\end{sol}

\begin{example}
	A rocket of total mass $m_0$ in empty space has speed $v_0$ when it must slow down to speed $\nicefrac{v_0}{2}$ to intercept an asteroid. how much fuel must be burned?
\end{example}

\begin{sol}
	In this case, the fuel must be ejected forward, so relative velocity of the fuel is $v + u$. We get an alter version

	\begin{equation}
		M_R \frac{\D v}{\D t} + u \frac{\D M_F}{\D t} = 0 \longrightarrow  m \frac{\D v}{\D t} = u \frac{\D m}{\D t}
	\end{equation}

	So we have

	\begin{align}
		m \D v &= u \D m\\
		\int_{m_0}^{m_f} \frac{\D m}{m} &= \int_{v_0}^{\nicefrac{v_0}{2}} \frac{\D v}{u}\\
		\log\left(\nicefrac{m_f}{m_0}\right) = - \frac{v_0}{2u}\\
		m_f &= m_0 e^{-\nicefrac{v_0}{2u}}
	\end{align}

	Amount of fuel is $m_0 - m_f$

	\begin{equation}
		\Delta m = m_0 \left(1 - e^{-\nicefrac{v_0}{2u}}\right)
	\end{equation}

	This is also known as the Tsiolkovsky rocket equation

	\begin{equation}
		m_0 = m_f e^{\nicefrac{\Delta v}{u}}
	\end{equation}
\end{sol}

\section{Impulse}

We know that $\vec F = \frac{\D \vec p}{\D t}$, then

\begin{equation}
	\Delta \vec p = \int_{t_0}^{t_f} \vec F \D t
\end{equation}

\begin{definition}
	We defined impulse to be

	\begin{equation}
		\vec J = \int_{t_0}^{t_f} \vec F \D t = \Delta \vec p
	\end{equation}
\end{definition}

\begin{remark}
	Consider $\vec F = \mathrm{const}$, then $\vec J = \vec F \Delta t$

	\begin{itemize}
		\item So large $\vec F$ over short time means a large $\vec J$
		\item A large $\Delta t$ and small $\vec F$ also means a large $\vec J$
	\end{itemize}
\end{remark}

\section{Conservation}

For a system of particles

\begin{equation}
	\vec F_\mathrm{ext} = \frac{\D \vec p}{\D t}
\end{equation}

If $\vec F_\mathrm{ext} = 0$, then $\nicefrac{\D \vec p}{\D t} = 0$, which means $\vec p$ is constant in time.

\begin{definition}[The Law of Conservation of Momentum]
	For a system of particles such that the total external force on the system is zero, then the total linear momentum $\vec P = \sum_i \vec p_i$ is conserved -- $\Delta \vec p = 0$.
\end{definition}

Recalled that energy $E$ is conserved b/c $u$ is time-independent. Supposed $u = u(y, z)$ that is, $u$ is independent of $x$.

Then,

\begin{equation}
	\vec F = - \nabla u = - \left[\frac{\partial u}{\partial x} \hat x + \frac{\partial u}{\partial y} \hat y + \frac{\partial u}{\partial z} \hat z\right] = \frac{\partial u}{\partial y} \hat y + \frac{\partial u}{\partial z} \hat z
\end{equation}

Hence,

\begin{equation}
	-\nabla u = \frac{\D \vec p}{\D t} \implies \begin{cases}
		0 &= \frac{\D \vec p_x}{\D t}\\
		- \frac{\partial u}{\partial y} &= \frac{\D p_y}{\D t}\\
		- \frac{\partial u}{\partial z} &= \frac{\D p_z}{\D t}
	\end{cases}
\end{equation}

\begin{itemize}
	\item If $u$ is space-translation invariant $\implies$ momentum is conserved.
	\item Because $\vec p$ is a vector, it may be that only some components are conserved.
\end{itemize}

\section{Collisions}

A collision is a short duration interaction between two objects, such that external forces are negligible in comparison to internal forces of interaction over that duration and hence momentum is conserved for this interaction.

We will classify collisions into three categories, based on the kinetic energy difference. Let $Q \equiv K_0 - K_f$ (total K.E.)

\begin{description}
	\item[Q > 0] Inelastic Collision
	
	Kinetic energy is ``lost'' to other forms.

	\textit{If particles stick together after collision, we have \textbf{perfectly inelastic collision}}

	\item[Q = 0] Elastic Collision

	Kinetic energy is conserved.

	*note, typically the internal interaction is conservative, so $KE_0 \to PE \to KE_f$, meaning energy conserved.

	\item[Q < 0] Superelastic Collision
	
	Kinetic energy is gained (i.e. an explosion)
\end{description}

\begin{example}
	Two particles masses $m_1$ and $m_2$ with initial velocities $\vec v_1$ and $\vec v_2$ collide in a perfectly inelastic collision. Fidn the velocity $\vec v'$ of the combined mass after the collision and the kinetic energy lost.
\end{example}

\begin{sol}
	Conservation of momentum:

	\begin{align}
		\vec p &= \vec p'\\
		m_1 \vec v_1 + m_2 \vec v_2 &= (m_1 + m_2) \vec v'\\
		v' = \boxed{\frac{m_1 \vec v_1 + m_2 \vec v_2}{m_1 + m_2}}
	\end{align}

	Let's find $Q$:

	\begin{align}
		Q &= K - K'\\
		&= \frac{1}{2} m_1 {v_1}^2 + \frac{1}{2} m_2 {v_2}^2 - \frac{1}{2} (m_1 + m_2) {v'}^2\\
		&= \frac{1}{2} \left[m_1 - \frac{{m_1}^2}{m_1 + m_2}\right] {v_1}^2 + \frac{1}{2} \left[m_2 - \frac{{m_2}^2}{m_1 + m_2}\right] {v_2}^2 - \left(\frac{m_1 m_2}{m_1 + m_2}\right) \vec v_1 \cdot \vec v_2\\
		&= \boxed{\frac{1}{2} \left(\frac{m_1 m_2}{m_1 + m_2}\right) (\vec v_1 - \vec v_2)^2}
	\end{align}

	There are several cases

	\begin{itemize}
		\item If $v_1 = v_2$, they never collide!
		\item If $v_1 = - v_2$, they collide with energy lost, which is the total energy.
	\end{itemize}
\end{sol}